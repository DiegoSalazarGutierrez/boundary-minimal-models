\documentclass[a4paper, 12pt, reqno]{amsart}

\usepackage{amssymb}
\usepackage[backref = page]{hyperref}
\usepackage[margin = 0.75in]{geometry}
\usepackage[shortlabels]{enumitem}
\usepackage{tikz-cd}
\usepackage[nameinlink]{cleveref}

\newtheorem{theorem}{Theorem}[section]
\newtheorem{lemma}[theorem]{Lemma}
\newtheorem{proposition}[theorem]{Proposition}
\newtheorem{corollary}[theorem]{Corollary}

\theoremstyle{remark}
\newtheorem{remark}[theorem]{Remark}
\newtheorem{example}[theorem]{Example}

\crefformat{section}{\S#2#1#3}

\setenumerate[0]{label = \normalfont(\roman*)}

\DeclareMathOperator{\Vir}{Vir}
\DeclareMathOperator{\Ind}{Ind}
\DeclareMathOperator{\ch}{ch}
\DeclareMathOperator{\len}{len}
\DeclareMathOperator{\clen}{clen}
\DeclareMathOperator{\vac}{|0\rangle}
\DeclareMathOperator{\gr}{gr}
\DeclareMathOperator{\Hom}{Hom}
\DeclareMathOperator{\vspan}{span}
\DeclareMathOperator{\supp}{supp}
\DeclareMathOperator{\Frac}{Frac}
\DeclareMathOperator{\ad}{ad}
\DeclareMathOperator{\lp}{lp}
\DeclareMathOperator{\prt}{prt}
\DeclareMathOperator{\res}{res}
\DeclareMathOperator{\Der}{Der}
\DeclareMathOperator{\sing}{sing}
\DeclareMathOperator{\End}{End}
\DeclareMathOperator{\Id}{Id}
\DeclareMathOperator{\fs}{fs}
\DeclareMathOperator{\inc}{inc}

\renewcommand*{\backref}[1]{}
\renewcommand*{\backrefalt}[4]{%
  \ifcase #1 (Not cited.)%
  \or        (Cited on page~#2.)%
  \else      (Cited on pages~#2.)%
  \fi}

\begin{document}

\setcounter{section}{-1}

\begin{abstract}
  Some abstract. \\
  \smallskip
  \noindent \textbf{Keywords.} Some keywords.
\end{abstract}

\title{Boundary minimal models}
\author{Diego Salazar}
\address{Instituto de Matemática Pura e Aplicada, Rio de Janeiro, RJ, Brazil}
\email{diego.salazar@impa.br}
\date{\today}
\maketitle

\section{Introduction}
\label{sec:introduction}

Let $s \in \mathbb{Z}_+$, let $i \in \{1, \dots, s\}$, let $R^{s, i}$ denote the following set of partitions
\begin{equation*}
  R^{s, i} = \{\text{$\lambda$ is a partition} \mid \text{$\lambda = [\lambda_1, \dots, \lambda_s]$ satisfies $\lambda_1 - \lambda_s \le 1$ or $\lambda = \underbrace{[1, 1, \dots, 1]}_i$}\},
\end{equation*}
and let $P^{s, i}$ be the set of partitions that do not contain any partition in $R^{s, i}$.

\begin{theorem}
  \label{thr:1}
  The refined character of $L(c_{2, 2s + 1}, h_{1, i})$ is given by
  \begin{equation*}
    \ch_{L(c_{2, 2s + 1}, h_{1, i})}(t, q) = h^{1, i}J_{s, i}(0, t, q) = h^{1, i}\left(\sum_{k = (k_1, \dots, k_{s - 1}) \in \mathbb{N}^{s - 1}}t^{kB^{(s)}_{s - 1}}\frac{q^{\frac{1}{2}kG^{(s)}k^T + kB^{(s)}_{s - i}}}{(q)_{k_1}\dots(q)_{k_{s - 1}}}\right),
  \end{equation*}
  where $T$ denotes the transpose matrix and
  \begin{align*}
    G^{(s)} &= (2\min\{i, j\})_{i, j = 1}^{s - 1}, \\
    B^{(s)}_j &=
                \left(\begin{smallmatrix}
                  0 \\
                  0 \\
                  \vdots \\
                  0 \\
                  1 \\
                  2 \\
                  \vdots \\
                  j
                \end{smallmatrix}\right) \quad \text{for $j = 0, 1, \dots, s - 1$}.
  \end{align*}
\end{theorem}

\begin{theorem}
  \label{thr:2}
  The set
  \begin{equation*}
    \{L_{-\lambda_1}L_{-\lambda_2}\dots L_{-\lambda_m}|c_{2, 2s + 1}, h_{1, i}\rangle \mid \lambda = [\lambda_1, \dots, \lambda_m] \in P^{s, i}\}
  \end{equation*}
  is a vector space basis of $L(c_{2, 2s + 1}, h_{1, i})$.
\end{theorem}

\begin{theorem}
  \label{thr:3}
  The boundary minimal models $L(c_{2, 2s + 1}, h_{1, i})$ are classically free for $s \in \mathbb{Z}_+$ and $i \in \{1, \dots, s\}$.
\end{theorem}

\section{Preliminaries and notation}
\label{sec:prel-notat}

All vector spaces and all algebras are over $\mathbb{C}$, the field of complex numbers, unless otherwise stated.
All tensor products are over $\mathbb{C}$, unless otherwise stated.
The set of natural numbers $\{0, 1, \dots\}$ is denoted by $\mathbb{N}$, the set of integers is denoted by $\mathbb{Z}$, and the set of positive integers $\{1, 2, \dots\}$ is denoted by $\mathbb{Z}_+$.

First, we review the theory of representations of the Virasoro Lie algebra following \cite{kac_bombay_2013}.
The \emph{Virasoro Lie algebra} is a Lie algebra given by
\begin{equation*}
  \Vir = \bigoplus_{n \in \mathbb{Z}}\mathbb{C}L_n \oplus \mathbb{C}C.
\end{equation*}
These elements satisfy the following commutation relations:
\begin{equation}
  \label{eq:1}
  \begin{split}
    [L_m, L_n] &= (m - n)L_{m + n} + \delta_{m, -n}\frac{m^3 - m}{12}C \quad \text{for $m, n \in \mathbb{Z}$}, \\
    [\Vir, C] &= 0.
  \end{split}
\end{equation}
Let $c, h \in \mathbb{C}$.
We set $\Vir^{\ge} = \bigoplus_{n \in \mathbb{N}}\mathbb{C}L_n \oplus \mathbb{C}C$.
We make the subalgebra $\Vir^{\ge}$ of $\Vir$ act on $\mathbb{C}$ as follows:
\begin{equation*}
  \text{$L_n1 = 0$ for $n \in \mathbb{Z}_+$, $L_01 = h$ and $C1 = c$}.
\end{equation*}
The \emph{Verma representation} of $\Vir$ with \emph{highest weight} $(c, h)$ is defined as
\begin{equation*}
  M(c, h) = \Ind^{\Vir}_{\Vir^{\ge}}(\mathbb{C}) = U(\Vir) \otimes_{U(\Vir^{\ge})} \mathbb{C},
\end{equation*}
where $\Vir$ acts by left multiplication.
We take $|c, h\rangle = 1 \otimes 1$ as \emph{highest weight vector}.

A \emph{composition (of $n \in \mathbb{N}$)} is a sequence $\lambda = [\lambda_1, \dots, \lambda_m]$ such that $\lambda_i \in \mathbb{Z}_+$ for $i = 1, \dots, m$ (and $\lambda_1 + \dots + \lambda_m = n$).
We also consider the \emph{empty composition} $\emptyset$, which is the unique composition of $0$.
Given a composition $\lambda = [\lambda_1, \dots, \lambda_m]$ and a permutation $\sigma \in S_m$, we define the composition
\begin{equation*}
  \lambda\sigma = [\lambda_{\sigma(1)}, \dots, \lambda_{\sigma(m)}].
\end{equation*}

For a composition $\lambda = [\lambda_1, \dots, \lambda_m]$, we define
\begin{equation*}
  L_{\lambda} = L_{-\lambda_1}\dots L_{-\lambda_m} \in U(\Vir),
\end{equation*}
the \emph{length of $\lambda$} as
\begin{equation*}
  \len(\lambda) = m,
\end{equation*}
the \emph{weight of $\lambda$} as
\begin{equation*}
  \Delta(\lambda) = \lambda_1 + \dots + \lambda_m,
\end{equation*}
and
\begin{equation*}
  p_{\lambda} = L_{-\lambda_1}\dots L_{-\lambda_m} \in \mathbb{C}[L_{-1}, L_{-2}, \dots].
\end{equation*}

For a composition $\lambda$ with exactly $m$ elements greater than $1$ and exactly $n$ elements equal to $1$, let $\lambda'$ be the composition obtained from $\lambda$ by removing all $1$s.
We define the \emph{conformal length of $\lambda$} as
\begin{equation*}
  \clen(\lambda) = 2m + n
\end{equation*}
and
\begin{equation*}
  u_{\lambda} = p_{\lambda'}L_{-1}^n \in \bigoplus_{k \in \mathbb{N}}\mathbb{C}[L_{-2}, L_{-3}, \dots]L_{-1}^k.
\end{equation*}

A \emph{partition (of $n \in \mathbb{N}$)} is a composition $\lambda = [\lambda_1, \dots, \lambda_m]$ (of $n$) such that $\lambda_1 \ge \dots \ge \lambda_m$.
The notation $\lambda \vdash$ means $\lambda$ is a partition, and for $n \in \mathbb{N}$, the notation $\lambda \vdash n$ means $\lambda$ is a partition of $n$.
This notation also tacitly assumes that $\lambda$ is a partition and not just a composition.

Given a composition $\lambda$ of length $m$, there is a unique partition $\prt(\lambda)$ obtained by reordering $\lambda$, i.e., by considering all $\lambda\sigma$ for $\sigma \in S_m$.

By the PBW theorem, the set
\begin{equation*}
  \{L_{\lambda}|c, h\rangle \mid \lambda \vdash\}
\end{equation*}
is a vector space basis of $M(c, h)$.
The representation $M(c, h)$ has a unique maximal proper subrepresentation $J(c, h)$, and the quotient
\begin{equation*}
  L(c, h) = M(c, h)/J(c, h)
\end{equation*}
is the \emph{irreducible highest weight representation} of $\Vir$ with highest weight $(c, h)$.

\section{Modules over algebras with a filtration}
\label{sec:modul-over-algebr}

Let $A$ be an associative (not necessarily commutative) algebra with unit $1$ and filtration $(A^p)_{p \in \mathbb{Z}}$ such that:
\begin{enumerate}
\item $A^p = 0$ for $p < 0$;
\item $1 \in A^0$;
\item $A^0 \subseteq A^1 \subseteq \dots$;
\item $A^pA^q \subseteq A^{p + q}$ for $p, q \in \mathbb{Z}$.
\end{enumerate}
Let
\begin{equation*}
  \gr(A) = \bigoplus_{p \in \mathbb{N}}A^p/A^{p - 1}
\end{equation*}
be the associated graded vector space.
The vector space $\gr(A)$ is an associative algebra with unit and multiplication given as follows.
For $p, q \in \mathbb{N}$, $a \in A^p$ and $b \in A^p$, we set
\begin{equation*}
  \gamma^p(a)\gamma^q(b) = \gamma^{p + q}(ab),
\end{equation*}
where $\gamma^p: A^p \to \gr(A)$ is the \emph{principal symbol map}, which is the composition of the natural maps $A^p \twoheadrightarrow A^p/A^{p - 1}$ and $A^p/A^{p - 1} \hookrightarrow \gr(A)$.
The unit of $\gr(A)$ is $\gamma^0(1)$.

Let $\partial$ be a derivation of $A$ respecting the grading $(A^p)_{p \in \mathbb{Z}}$, i.e., it satisfies:
\begin{enumerate}
\item $\partial(ab) = \partial(a)b + a\partial(b)$ for $a, b \in A$;
\item $\partial(A^p) \subseteq A^p$ for $p \in \mathbb{Z}$.
\end{enumerate}
We can define
\begin{align*}
  \partial: \gr(A) &\to \gr(A), \\
  \partial(\gamma^p(a)) &= \gamma^p(\partial(a)) \quad \text{for $p \in \mathbb{Z}$ and $a \in A^p$},
\end{align*}
and it is a derivation of $\gr(A)$.

\begin{example}[\emph{PBW filtration of $U(\mathfrak{g})$}]
  \label{exa:1}
  Let $\mathfrak{g}$ be a Lie algebra.
  The PBW filtration of $U(\mathfrak{g})$, the universal enveloping algebra of $\mathfrak{g}$, is given by
  \begin{equation*}
    U(\mathfrak{g})^p = \vspan\{x_1x_2\dots x_s \mid s \le p, x_1, \dots, x_s \in \mathfrak{g}\} \quad \text{for $p \in \mathbb{Z}$}.
  \end{equation*}
  This filtration clearly satisfies axioms (i)--(iv) above, and $\gr(U(\mathfrak{g}))$ is naturally isomorphic to $S(\mathfrak{g})$, the symmetric algebra of $\mathfrak{g}$, which is a polynomial algebra (see \cite[\S2]{dixmier_enveloping_1996} for details).
  Furthermore, if $\partial$ is a derivation of $\mathfrak{g}$ as a Lie algebra, then we can extend $\partial$ to a derivation of $U(\mathfrak{g})$, and it respects the PBW filtration.
  Thus, it defines a derivation $\partial: \gr(U(\mathfrak{g})) \to \gr(U(\mathfrak{g}))$.
\end{example}

Let $M$ be an $A$-module with filtration $(M^p)_{p \in \mathbb{Z}}$ such that:
\begin{enumerate}
\item $M^p = 0$ for $p < 0$;
\item $M^0 \subseteq M^1 \subseteq \dots$;
\item $A^pM^q \subseteq M^{p + q}$ for $p, q \in \mathbb{Z}$.
\end{enumerate}
Let
\begin{equation*}
  \gr(M) = \bigoplus_{p \in \mathbb{N}}M^p/M^{p - 1}
\end{equation*}
be the associated graded vector space.

Then $\gr(M)$ is a $\gr(A)$-module with operations given as follows.
For $p, q \in \mathbb{N}$, $a \in A^p$ and $u \in M^p$, we set
\begin{equation*}
  \gamma^p(a)\gamma^q_M(u) = \gamma^{p + q}_M(au),
\end{equation*}
where $\gamma^p_M: M^p \to \gr(M)$ is the principal symbol map, which is the composition of the natural maps $M^p \twoheadrightarrow M^p/M^{p - 1}$ and $M^p/M^{p - 1} \hookrightarrow \gr(M)$.

The category of modules over $A$ with the given filtration $(A^p)_{p \in \mathbb{Z}}$ is given by modules $M$ with filtration $(M^p)_{p \in \mathbb{Z}}$ satisfying conditions (i)--(iii) above.
A homomorphism $f: M \to N$ must satisfy $f(M^p) \subseteq N^p$ for $p \in \mathbb{Z}$.
This category is denoted by $A$-Mod (the filtration being suppressed from the notation).

If $f: M \to N$ is a homomorphism of $A$-modules, then
\begin{align*}
  \gr(f): \gr(M) &\to \gr(N), \\
  \gr(f)(\gamma_M^p(u)) &= \gamma_N^p(f(u)) \quad \text{for $p \in \mathbb{N}$ and $u \in M^p$}
\end{align*}
defines a homomorphism of $\gr(A)$-modules.
Therefore, we obtain a functor
\begin{equation*}
  \gr: \{\text{$A$-modules}\} \to \{\text{$\gr(A)$-modules}\}.
\end{equation*}

From now on, some subscripts or superscripts will be omitted, so for example $\gamma^p_M$ simplifies to $\gamma^p$.

\section{Modules over $Q$-graded Lie algebras}
\label{sec:modules-over-q}

In this section, we follow \cite[\S2]{iohara_representation_2011}.
Let $\Gamma$ be an abelian group.
For a $\Gamma$-graded vector space $V = \bigoplus_{\alpha \in \Gamma}V^{\alpha}$, we set $\supp(V) = \{\alpha \in \Gamma \mid V^{\alpha} \neq 0\}$.

A \emph{$\Gamma$-graded Lie algebra} is a Lie algebra $\mathfrak{g} = \bigoplus_{\alpha \in \Gamma}\mathfrak{g}^{\alpha}$ such that
\begin{equation*}
  [\mathfrak{g}^{\alpha}, \mathfrak{g}^{\beta}] \subseteq \mathfrak{g}^{\alpha + \beta} \quad \text{for $\alpha, \beta \in \Gamma$}.
\end{equation*}

Let $Q$ be a free abelian group of finite rank $r$, and let $\mathfrak{g}$ be a Lie algebra with a commutative subalgebra $\mathfrak{h}$.
We say a pair $(\mathfrak{g}, \mathfrak{h})$ is a \emph{$Q$-graded Lie algebra} if it satisfies the following:
\begin{enumerate}
\item $\mathfrak{g} = \bigoplus_{\alpha \in Q}\mathfrak{g}^{\alpha}$ is $Q$-graded, $\mathfrak{h} = \mathfrak{g}^0$, and $\supp(\mathfrak{g})$ generates $Q$;
\item We have a homomorphism $\pi_Q: Q \to \mathfrak{h}^*, \alpha \mapsto \lambda_{\alpha}$ such that
  \begin{equation*}
    [h, x] = \lambda_{\alpha}(h)x \quad \text{for $h \in \mathfrak{h}$ and $x \in \mathfrak{g}^{\alpha}$};
  \end{equation*}
\item For $\alpha \in Q$, $\dim(\mathfrak{g}^{\alpha}) < \infty$;
\item There exists a basis $(\alpha_i)_{i = 1}^r$ of $Q$ such that for $\alpha \in \supp(\mathfrak{g})$,
  \begin{equation*}
    \text{$\alpha \in \sum_{i = 1}^r\mathbb{Z}_{\ge 0}\alpha_i$ or $\alpha \in \sum_{i = 1}^r\mathbb{Z}_{\le 0}\alpha_i$}.
  \end{equation*}
\end{enumerate}

The condition (iv) implies that a $Q$-graded Lie algebra admits a \emph{triangular decomposition}
\begin{equation*}
  \mathfrak{g} = \mathfrak{g}^- \oplus \mathfrak{h} \oplus \mathfrak{g}^+,
\end{equation*}
where $Q^+ = \sum_{i = 1}^r\mathbb{Z}_{\ge 0}\alpha_i$ and
\begin{equation*}
  \mathfrak{g}^{\pm} = \bigoplus_{\pm \alpha \in Q^+ \setminus \{0\}}\mathfrak{g}^{\alpha}.
\end{equation*}
For later use, we set $\mathfrak{g}^{\ge} = \mathfrak{h} \oplus \mathfrak{g}^+$ and $\mathfrak{g}^{\le} = \mathfrak{g}^- \oplus \mathfrak{h}$.

\begin{example}
  \label{exa:2}
  We set $Q = \mathbb{Z}$, $\mathfrak{h} = \mathbb{C}L_0 \oplus \mathbb{C}C$, $\Vir^n = \mathbb{C}L_n$ for $n \in \mathbb{Z} \setminus \{0\}$ and
  \begin{align*}
    \pi_Q: Q &\to \mathfrak{h}^*, \\
    \pi_Q(n) &= (L_0 \mapsto -n, C \mapsto 0).
  \end{align*}
  Then $(\Vir, \mathfrak{h})$ is readily seen to be a $\mathbb{Z}$-graded Lie algebra.
  We take $1$ as the basis of $Q$, so $Q^+ = \mathbb{N}$.
  We can verify that $\ad(L_{-1})$ is a derivation of $\Vir$ satisfying $\ad(L_{-1})(\Vir^-) \subseteq \Vir^-$.
  Therefore, $\gr(\Vir^-) \cong \mathbb{C}[L_{-1}, L_{-2}, \dots]$ has a derivation given by $\partial(L_{-n}) = (n - 1)L_{-n - 1}$ for $n \in \mathbb{Z}_+$.
\end{example}

\begin{example}[\emph{(Untwisted) affinization of $\mathfrak{g}$}]
  \label{exa:3}
  Let $\mathfrak{g}$ be a finite dimensional Lie algebra over $\mathbb{C}$, let $\mathfrak{h}$ be a Cartan subalgebra of $\mathfrak{g}$, and let $(\bullet, \bullet)$ be a nondegenerate invariant bilinear form on $\mathfrak{g}$.
  We set:
  \begin{align*}
    \bar{\mathfrak{g}} &= \mathfrak{g} \otimes \mathbb{C}[t, t^{-1}] \oplus \mathbb{C}K \oplus \mathbb{C}d, \\
    [xt^m, yt^n] &= [x, y]t^{m + n} + m\delta_{m, -n}(x, y)K \quad \text{for $x, y \in \mathfrak{g}$ and $m, n \in \mathbb{Z}$}, \\
    [K, \bar{\mathfrak{g}}] &= 0, \\
    [d, xt^m] &= mxt^m \quad \text{for $x \in \mathfrak{g}$ and $m \in \mathbb{Z}$}.
  \end{align*}
  With this bracket, $\bar{\mathfrak{g}}$ becomes a Lie algebra called the \emph{(untwisted) affine Lie algebra of $\mathfrak{g}$} or the (untwisted) affinization of $\mathfrak{g}$.
  If $\mathfrak{g}$ is simple, $\bar{\mathfrak{g}}$ is called the \emph{(untwisted) Kac-Moody affinization of $\mathfrak{g}$}.

  Let $\Delta \subseteq \mathfrak{h}^*$ be the set of roots of $\mathfrak{g}$ with respect to $\mathfrak{h}$, so we have a root space decomposition $\mathfrak{g} = \mathfrak{h} \bigoplus_{\beta \in \Delta}\mathfrak{g}^{\beta}$, where $\mathfrak{g}^{\beta} = \{x \in \mathfrak{g} \mid \text{for $h \in \mathfrak{h}$, $[h, x] = \beta(h)x$}\}$ for $\beta \in \mathfrak{h}^*$.
  We fix a set of simple roots $(\alpha_i)_{i = 1}^r$ of $\mathfrak{g}$ and denote the highest root by $\theta$.

  We set
  \begin{equation*}
    \bar{\mathfrak{h}} = \mathfrak{h}t^0 \oplus \mathbb{C}K \oplus \mathbb{C}d
  \end{equation*}
  and for $\beta \in \mathfrak{h}^*$, we extend $\beta$ by defining $\bar{\beta} \in \bar{\mathfrak{h}}^*$ as follows
  \begin{align*}
    \bar{\beta}: \bar{\mathfrak{h}}^* &\to \mathbb{C}, \\
    \bar{\beta}(ht^0) &= \beta(h) \quad \text{for $h \in \mathfrak{h}$}, \\
    \bar{\beta}(K) = \bar{\beta}(d) &= 0.
  \end{align*}
  We get a new set of roots $\bar{\Delta} \subseteq \bar{\mathfrak{h}}^*$ given by
  \begin{equation*}
    \bar{\Delta} = \{\bar{\beta} \mid \beta \in \Delta\},
  \end{equation*}
  and we define one more root $\delta \in \bar{\mathfrak{h}}^*$ by setting
  \begin{align*}
    \delta: \bar{\mathfrak{h}} &\to \mathbb{C}, \\
    \delta(\mathfrak{h}t^0) &= 0, \\
    \delta(K) &= 0, \\
    \delta(d) &= 1.
  \end{align*}
  Finally, we set
  \begin{equation*}
    Q = \mathbb{Z}\bar{\Delta} \oplus \mathbb{Z}\delta \subseteq \bar{\mathfrak{h}}^*
  \end{equation*}
  and for $\alpha \in Q$,
  \begin{equation*}
    \bar{\mathfrak{g}}^{\alpha} =
    \begin{cases}
      \mathfrak{g}^{\beta}t^n &\text{if $\alpha = \bar{\beta} + n\delta$ for some $\bar{\beta} \in \bar{\Delta}$ and $n \in \mathbb{Z}$}; \\
      \mathfrak{h}t^n &\text{if $\alpha = n\delta$ for some $n \in \mathbb{Z} \setminus \{0\}$}; \\
      \mathfrak{h} &\text{if $\alpha = 0$}; \\
      0 &\text{otherwise}.
    \end{cases}
  \end{equation*}
  Then $\bar{\mathfrak{g}} = \bigoplus_{\alpha \in Q}\bar{\mathfrak{g}}^{\alpha}$ is the root space decomposition with respect to $\bar{\mathfrak{h}}$, and $\Pi = \{\alpha_1, \dots, \alpha_r\} \cup \{\alpha_0\}$ is a $\mathbb{Z}$-basis of $Q$ satisfying (iv), where $\pi_Q$ is the inclusion map and $\alpha_0 = \delta - \theta$.
  Hence, $(\mathfrak{g}, \mathfrak{h})$ is a $Q$-graded Lie algebra.
\end{example}

\begin{example}[\emph{Affinization of $\mathfrak{g}$}]
  \label{exa:4}
  With notation as in the last example, we set
  \begin{equation*}
    \hat{\mathfrak{g}} = [\bar{\mathfrak{g}}, \bar{\mathfrak{g}}] = \mathfrak{g} \otimes \mathbb{C}[t, t^{-1}] \oplus \mathbb{C}K.
  \end{equation*}
  With this bracket, $\hat{\mathfrak{g}}$ becomes a Lie algebra called the \emph{affine Lie algebra of $\mathfrak{g}$} or the affinization of $\mathfrak{g}$.
  If $\mathfrak{g}$ is simple, $\hat{\mathfrak{g}}$ is called the \emph{Kac-Moody affinization of $\mathfrak{g}$}.

  We set
  \begin{equation*}
    \hat{h} = \bar{\mathfrak{h}} \cap \hat{\mathfrak{g}} = \mathfrak{h}t^0 \oplus \mathbb{C}K.
  \end{equation*}
  Then $(\hat{\mathfrak{g}}, \hat{\mathfrak{h}})$ is a $Q$-graded Lie algebra with $Q$-grading
  \begin{equation*}
    \hat{\mathfrak{g}}^{\alpha} =
    \begin{cases}
      \bar{\mathfrak{g}}^{\alpha} &\text{if $\alpha \neq 0$}; \\
      \hat{\mathfrak{h}} &\text{if $\alpha = 0$}.
    \end{cases}
  \end{equation*}
  In this case, the map $\pi_Q: Q \to (\mathfrak{h}')^*$ is not injective because $\pi_Q(\delta) = 0$.
\end{example}

Let $\Gamma$ be an abelian group, and let $\mathfrak{g} = \bigoplus_{\alpha \in \Gamma}\mathfrak{g}^{\alpha}$ be a $\Gamma$-graded Lie algebra.
A $\mathfrak{g}$-module $M = \bigoplus_{\alpha \in \Gamma}M^{\alpha}$ is \emph{$\Gamma$-graded} if
\begin{equation*}
  \mathfrak{g}^{\alpha}M^{\beta} \subseteq M^{\alpha + \beta} \quad \text{for $\alpha, \beta \in \Gamma$}.
\end{equation*}

An $\mathfrak{h}$-module $M$ is \emph{$\mathfrak{h}$-diagonalizable} if
\begin{equation*}
  M = \bigoplus_{\lambda \in \mathfrak{h}^*}M_{\lambda},
\end{equation*}
where
\begin{equation*}
  M_{\lambda} = \{u \in M \mid \text{for $h \in \mathfrak{h}$, $hu = \lambda(h)u$}\}.
\end{equation*}

An $\mathfrak{h}$-diagonalizable module $M$ is called \emph{$\mathfrak{h}$-semisimple} if
\begin{equation*}
  \dim(M_{\lambda}) < \infty \quad \text{for $\lambda \in \mathfrak{h}^*$}.
\end{equation*}
A \emph{$(\mathfrak{g}, \mathfrak{h})$-module} is an $\mathfrak{h}$-diagonalizable $\mathfrak{g}$-module.
From now on, we assume that $\pi_Q$ is injective.

The \emph{character of a semisimple $(\mathfrak{g}, \mathfrak{h})$-module $M$} is given by
\begin{equation*}
  \ch_M(q) = \sum_{\lambda \in \mathfrak{h}^*}\dim(M_{\lambda})q^{\lambda}.
\end{equation*}

We can regard $\mathfrak{g}$ as a $(\mathfrak{g}, \mathfrak{h})$-module via the adjoint representation and by declaring for $\lambda \in \mathfrak{h}^*$,
\begin{equation*}
  \mathfrak{g}_{\lambda} = \mathfrak{g}^{\alpha} \quad \text{if there exists $\alpha \in Q$ such that $\lambda = \pi_Q(\alpha)$}.
\end{equation*}
This construction assumes implicitly that $\pi_Q$ is injective.

The category $\mathcal{C}_{(\mathfrak{g}, \mathfrak{h})}$ is the category whose objects are $\mathfrak{h}^*$-graded $(\mathfrak{g}, \mathfrak{h})$-modules and for $M, N \in \mathcal{C}_{(\mathfrak{g}, \mathfrak{h})}$, $\Hom_{\mathcal{C}_{(\mathfrak{g}, \mathfrak{h})}}(M, N) = \Hom_{\mathfrak{g}}(M, N)$ (Lie algebra module homomorphisms).

\begin{remark}
  \label{rmk:1}
  If $f \in \Hom_{\mathfrak{g}}(M, N)$, then $f$ respects the $\mathfrak{h}^*$-grading automatically because for $u \in M$ and $h \in \mathfrak{h}$, $hf(u) = f(hu) = f(\lambda(h)u) = \lambda(h)f(u)$.
\end{remark}

We can introduce a partial order on $Q$ as follows
\begin{equation*}
  \lambda_1 \le \lambda_2 \iff \text{there exists $\gamma \in Q^+$ such that $\lambda_2 - \lambda_1 = \pi_Q(\gamma)$.}
\end{equation*}
Since $\pi_Q$ is assumed to be injective, we can abuse notation and write $\lambda_2 - \lambda_1 = \gamma \in Q$.

We now define the full subcategory $\mathcal{C}^{\lambda \ge}_{(\mathfrak{g}, \mathfrak{h})}$ by considering the objects $M \in \mathcal{C}_{(\mathfrak{g}, \mathfrak{h})}$ such that $M_{\mu} = 0$ if $\lambda \ngeq \mu$.

Let $(\mathfrak{g}, \mathfrak{h})$ be a $Q$-graded Lie algebra, let $M \in \mathcal{C}_{(\mathfrak{g}, \mathfrak{h})}$, and let $\lambda \in \mathfrak{h}^*$.
$M$ is called a \emph{highest weight module} with \emph{highest weight} $\lambda \in \mathfrak{h}^*$ if there is a nonzero vector $v \in M_{\lambda}$ such that:
\begin{enumerate}
\item $xv = 0$ for $x \in \mathfrak{g}^+$;
\item $U(\mathfrak{g}^-)v = M$.
\end{enumerate}
The vector $v$ is called a \emph{highest weight vector of $M$} and is unique up to multiplication by a nonzero scalar.

\begin{example}
  \label{exa:5}
  When considering the $\mathbb{Z}$-graded Virasoro Lie algebra $(\Vir, \mathfrak{h})$, where $\mathfrak{h} = \mathbb{C}L_0 \oplus \mathbb{C}C$, we identify $\mathfrak{h}^*$ with $\mathbb{C}^2$ as $\lambda = (c, h)$ if $\lambda(C) = c$ and $\lambda(L_0) = h$.
  We are mainly interested in the highest weight modules $M(c, h)$ and $L(c, h)$ of the Virasoro Lie algebra.
  In this case, $(c_2, n_2) \ge (c_1, n_1)$ if and only if $c_1 = c_2$ and $n_1 \ge n_2$.
  If this happens, $(c_2, n_2) - (c_1, n_1) = n_1 - n_2$.
\end{example}

\section{The refined character}
\label{sec:refined-character}

Let $(\mathfrak{g}, \mathfrak{h})$ be a $\mathbb{Z}$-graded Lie algebra, let $\lambda \in \mathfrak{h}^*$, and let $M \in \mathcal{C}^{\lambda \ge}_{(\mathfrak{g}, \mathfrak{h})}$.
The \emph{PBW filtration of $M$} is given by
\begin{equation*}
  M^p = \{au \mid q \in \mathbb{N}, \mu \le \lambda, a \in U(\mathfrak{g}^-)^q, u \in M_{\mu}, q + \lambda - \mu \le p\} \quad \text{for $p \in \mathbb{Z}$}.
\end{equation*}
This filtration clearly satisfies axioms (i)--(iii) of \Cref{sec:modul-over-algebr} with $U(\mathfrak{g}^-)$ in place of $A$, and $\gr(M)$ becomes a $\gr(U(\mathfrak{g^-}))$-module.

If $f: M \to N$ is a homomorphism, where $M, N \in \mathcal{C}^{\lambda \ge}_{(\mathfrak{g}, \mathfrak{h})}$, then
\begin{align*}
  \gr(f): \gr(M) &\to \gr(N), \\
  \gr(f)(\gamma_M^p(u)) &= \gamma_N^p(f(u)) \quad \text{for $p \in \mathbb{N}$ and $u \in M^p$}
\end{align*}
defines a homomorphism of $\gr(U(\mathfrak{g}^-))$-modules.
Therefore, we obtain a functor
\begin{equation*}
  \gr: \mathcal{C}^{\lambda \ge}_{(\mathfrak{g}, \mathfrak{h})} \to \gr(U(\mathfrak{g}^-))\text{-Mod}.
\end{equation*}

\begin{remark}
  \label{rmk:2}
  Let $M, N \in \mathcal{C}^{\lambda \ge}_{(\mathfrak{g}, \mathfrak{h})}$, and let $f: M \to N$ be a homomorphism.
  By \Cref{rmk:1}, $f$ respects the $\mathfrak{h}^*$-grading, and this implies $f(M^p) \subseteq N^p$, so what we wrote above makes sense.
\end{remark}

\begin{example}[$\gr(M(c, h))$]
  \label{exa:6}
  We pick a highest weight $(c, h)$.
  By \cite[\S2]{dixmier_enveloping_1996}, we have a natural isomorphism
  \begin{align*}
    \gr(M(c, h)) &\xrightarrow{\sim} \mathbb{C}[L_{-1}, L_{-2}, \dots] \cong \gr(U(\Vir^{-})), \\
    \gamma^{\len(\lambda)}(L_{\lambda}|c, h\rangle) & \mapsto p_{\lambda} \quad \text{for $\lambda$ a composition}.
  \end{align*}
\end{example}

The \emph{refined character of $M \in \mathcal{C}^{\lambda \ge}_{(\mathfrak{g}, \mathfrak{h})}$}, where $M$ is semisimple, is defined by
\begin{equation*}
  \ch_M(t, q) = \sum_{p \in \mathbb{N}}\sum_{\lambda \in \mathfrak{h}^*}\dim(\gamma^p(M^p \cap M_{\lambda}))t^pq^{\lambda},
\end{equation*}
and this is what we are interested in this article.
Clearly, we have
\begin{equation*}
  \ch_M(1, q) = \ch_M(q).
\end{equation*}

\begin{example}[$\ch_{M(c, h)}(t, q)$]
  \label{exa:7}
  We can consider the $\mathbb{Z}$-graded Virasoro Lie algebra $(\Vir, \mathfrak{h})$.
  We are mainly interested in $M(c, h)$ and $L(c, h)$ for some highest weights $(c, h)$.
  For the Verma representation $M(c, h)$, we have
  \begin{align*}
    \ch_{M(c, h)}(t, q) &= \frac{q^h}{\prod_{k \in \mathbb{Z}_+}(1 - tq^k)}, \\
    \ch_{M(c, h)}(q) &= \frac{q^h}{\prod_{k \in \mathbb{Z}_+}(1 - q^k)} = \frac{q^h}{(q)_{\infty}}.
  \end{align*}
\end{example}

\begin{example}[$\ch_{L(1/2, 0)}(t, q)$]
  \label{exa:8}
  In \cite{andrews_singular_2022}, the refined character of the Virasoro minimal model $\Vir_{3, 4} = L(1/2, 0)$, also known as the Ising model, is computed explicitly.
  Interestingly, this character is connected to Nahm sums for the matrix $\left(\begin{smallmatrix} 8 & 3 \\ 3 & 2 \end{smallmatrix}\right)$ (see \cite{Nahm2007}) and can be explicitly expressed as
  \begin{equation*}
    \ch_{L(1/2, 0)}(t, q) = \sum_{k_1, k_2 \in \mathbb{N}}t^{2k_1 + k_2}\frac{q^{4k_1^2 + 3k_1k_2 + k_2^2}}{(q)_{k_1}(q)_{k_2}}(1 - q^{k_1} + q^{k_1 + k_2}),
  \end{equation*}
  where $(q)_k = \prod_{j = 1}^k(1 - q^j) \in \mathbb{C}[q]$ denotes the \emph{$q$-Pochhammer symbol}.
\end{example}

\begin{remark}
  \label{rmk:3}
  Let $V$ be an $\mathbb{N}$-graded conformal vertex algebra, and let $M$ be an $h + \mathbb{N}$-graded $V$-module (see \Cref{sec:short-survey-vertex} ahead for the definitions).
  It is worth noting that the PBW filtration $(M^p)_{p \in \mathbb{Z}}$ and the standard filtration $(G^pM)_{p \in \mathbb{Z}}$, as introduced in \cite{salazar_pbw_2024}, are different.
  Therefore, the associated graded objects $\gr(M)$ and $\gr^G(M)$ and the refined characters $\ch_M(t, q)$ and $\ch_{\gr^G(M)}(t, q)$ are both different.
  For example, for a highest weight $(c, h)$,
  \begin{equation*}
    \ch_{\gr^G(M(c, h))}(t, q) = \frac{q^h}{(1 - tq)\prod_{k \ge 2}(1 - t^2q^k)}.
  \end{equation*}
\end{remark}

We pick a highest weight $(c, h)$.
We have a natural epimorphism
\begin{align*}
  \pi_{c, h}: M(c, h) &\twoheadrightarrow L(c, h), \\
  \pi_{c, h}(u) &= u + J(c, h),
\end{align*}
and it satisfies $\ker(\pi_{c, h}) = J(c, h)$.
Applying the functor $\gr$, we obtain an epimorphism of modules over $\gr(\Vir^-)$
\begin{equation*}
  \gr(\pi_{c, h}): \gr(M(c, h)) \twoheadrightarrow \gr(L(c, h)),
\end{equation*}
and this produces a natural isomorphism of modules over $\gr(\Vir^{-})$
\begin{equation*}
  \gr(M(c, h))/K(c, h) \xrightarrow{\sim} \gr(L(c, h)),
\end{equation*}
where
\begin{equation*}
  K(c, h) = \ker(\gr(\pi_{c, h})).
\end{equation*}
Explicitly, we have
\begin{equation}
  \label{eq:2}
  K(c, h) = \sum_{p \in \mathbb{N}}\gamma^p(J(c, h) \cap M(c, h)^p).
\end{equation}
From these observations, we obtain the following four isomorphisms which are going to be used frequently implicitly.

\begin{proposition}
  \label{prp:1}
  We pick a highest weight $(c, h)$.
  We have four (conformal) weight-preserving vector space isomorphisms:
  \begin{align*}
    M(c, h) &\xrightarrow{\sim} \gr(M(c, h)), \\
    L_{\lambda}|c, h\rangle &\mapsto \gamma^{\len(\lambda)}(L_{\lambda}|c, h\rangle), \\
    \gr(M(c, h)) &\xrightarrow{\sim} \mathbb{C}[L_{-1}, L_{-2}, \dots], \\
    \gamma^{\len(\lambda)}(L_{\lambda}|c, h\rangle) &\mapsto p_{\lambda}, \\
    L(c, h) &\xrightarrow{\sim} \gr(L(c, h)), \\
    L_{\lambda}(|c, h\rangle + J(c, h)) &\mapsto \gamma^{\len(\lambda)}(L_{\lambda}(|c, h\rangle + J(c, h))), \\
    \gr(L(c, h)) &\xrightarrow{\sim} \gr(M(c, h))/K(c, h), \\
    \gamma^{\len(\lambda)}(L_{\lambda}(|c, h\rangle + J(c, h))) &\mapsto \gamma^{\len(\lambda)}(L_{\lambda}|c, h\rangle) + K(c, h),
  \end{align*}
  where $\lambda$ is a partition.
\end{proposition}

In this article, we will deal with the polynomial algebra $\mathbb{C}[L_{-1}, L_{-2}, \dots]$, its ideals and quotients.
We will always use the degree reverse lexicographic order with $L_{-1} > L_{-2} > \dots$.
Actually, we can only use Gröbner basis theory with finitely generated polynomial algebras like $\mathbb{C}[L_{-1}, L_{-2}, \dots, L_{-N}]$ for some $N \in \mathbb{N}$.
But taking $N \to \infty$, we can obtain our desired results, see \cite{salazar_pbw_2024} for details regarding this.

\section{A short survey on vertex algebras}
\label{sec:short-survey-vertex}

The vector space of \emph{formal distributions in $n \in \mathbb{N}$ variables}, denoted by $\mathbb{C}[[x_1^{\pm1}, \dots, x_n^{\pm1}]]$, is the set of functions $f: \mathbb{Z}^n \to \mathbb{C}$, written as $f(x_1, \dots, x_n) = \sum_{m_1, \dots, m_n \in \mathbb{Z}}f_{m_1, \dots, m_n}x_1^{m_1}\dots x_n^{m_n}$, with the natural operations of addition and multiplication by a scalar.
The field of \emph{formal Laurent series}, denoted by $\mathbb{C}((x))$, is the subspace of elements $f(x) \in \mathbb{C}[[x^{\pm1}]]$ such that there is $N \in \mathbb{Z}$ with $f_n = 0$ for $n \le N$.
We also have $\mathbb{C}((x)) = \Frac(\mathbb{C}[[x]])$, so $\mathbb{C}((x))$ is actually a field.
If $V$ is a vector space, we similarly define $V[[x_1^{\pm1}, \dots, x_n^{\pm1}]]$ and $V((x))$, but in this case, $V((x))$ is only a vector space.
We can consider $V[[x_1^{\pm1}, \dots, x_n^{\pm1}]]$ a module over the polynomial algebra $\mathbb{C}[x_1, \dots, x_n]$.

Let $V$ be a vector space.
The \emph{Fourier expansion of a formal distribution $a(z) \in V[[z^{\pm1}]]$}, written as $a(z) = \sum_{n \in \mathbb{Z}}a_nz^n$, is conventionally written in the theory of vertex algebras as
\begin{equation*}
  a(z) = \sum_{n \in \mathbb{Z}}a_{(n)}z^{-n - 1},
\end{equation*}
where
\begin{equation*}
  a_{(n)} = a_{-n - 1}.
\end{equation*}
The \emph{residue of a formal distribution $a(z) \in V[[z^{\pm1}]]$} is defined as
\begin{equation*}
  \res_z(a(z)) = a_{(0)} = a_{-1}.
\end{equation*}

Let $V$ be a vector space.
A formal distribution $a(z, w) \in V[[z^{\pm1}, w^{\pm1}]]$ is \emph{local} if there is $N \in \mathbb{N}$ such that
\begin{equation*}
  (z - w)^Na(z, w) = 0.
\end{equation*}

Let $\mathfrak{g}$ be a Lie algebra.
A subset $\mathfrak{F} \subseteq \mathfrak{g}[[z^{\pm1}]]$ of formal distributions is called a \emph{local family} if all pairs of its elements are local.
For $n \in \mathbb{N}$, the \emph{$n$-product} is the $\mathbb{C}$-bilinear map defined by
\begin{align*}
  \bullet_{(n)}\bullet: \mathfrak{g}[[w^{\pm1}]] \times \mathfrak{g}[[w^{\pm1}]] &\to \mathfrak{g}[[w^{\pm1}]], \\
  a(w)_{(n)}b(w) &= \res_z((z - w)^n[a(z), b(w)]).
\end{align*}

Let $\mathfrak{g}$ be a Lie algebra.
A \emph{formal distribution Lie algebra} is a pair $(\mathfrak{g}, \mathfrak{F})$, where $\mathfrak{F}$ is a local family of $\mathfrak{g}$-valued formal distributions, denoted by $\{a^j(z) = \sum_{n \in \mathbb{Z}}a^j_{(n)}z^{-n - 1}\}_{j \in J}$, such that the coefficients $\{a^j_{(n)} \mid j \in J, n \in \mathbb{Z}\}$ span the whole $\mathfrak{g}$.
A \emph{regular} formal distribution Lie algebra is a triple $(\mathfrak{g}, \mathfrak{F}, T)$ such that:
\begin{enumerate}
\item $(\mathfrak{g}, \mathfrak{F})$ is a formal distribution Lie algebra;
\item $\mathbb{C}[\partial_z]\mathfrak{F}$ is closed under all $n$-th products for $n \in \mathbb{N}$;
\item $T \in \Der(\mathfrak{g})$ satisfies
  \begin{equation*}
    T(a^j(z)) = \partial_za^j(z) \quad \text{for $j \in J$}.
  \end{equation*}
\end{enumerate}

Let $(\mathfrak{g}, \mathfrak{F})$ be a formal distribution Lie algebra.
The \emph{annihilation subalgebra of $(\mathfrak{g}, \mathfrak{F})$} is
\begin{equation*}
  \mathfrak{g}_- = \vspan\{a^j_{(n)} \mid j \in J, n \in \mathbb{N}\},
\end{equation*}
the \emph{creation subalgebra of $(\mathfrak{g}, \mathfrak{F})$} is
\begin{equation*}
  \mathfrak{g}_+ = \vspan\{a^j_{(-n - 1)} \mid j \in J, n \in \mathbb{N}\},
\end{equation*}
and the \emph{polar decomposition of $(\mathfrak{g}, \mathfrak{F})$} is
\begin{equation*}
  \mathfrak{g} = \mathfrak{g}_- \oplus \mathfrak{g}_+.
\end{equation*}

\begin{example}
  \label{exa:9}
  We construct a $\Vir$-valued formal distribution by setting
  \begin{equation*}
    \text{$L(z) = \sum_{n \in \mathbb{Z}}L_{(n)}z^{-n - 1}$ with $L_{(n)} = L_{n - 1}$ for $n \in \mathbb{Z}$}.
  \end{equation*}
  We usually write $L(z)$ as
  \begin{equation*}
    L(z) = \sum_{n \in \mathbb{Z}}L_nz^{-n - 2}.
  \end{equation*}
  By \cite[\S2]{frenkel_vertex_2001}, $\{L(z), C\}$ is a local family.
  Therefore, $(\Vir, \{L(z), C\})$ is a formal distribution Lie algebra.
  Moreover, we can verify directly that $(\Vir, \{L(z), C\}, \ad(L_{-1}))$ is regular.
\end{example}

\begin{example}
  \label{exa:10}
  Let $\mathfrak{g}$ be a finite dimensional Lie algebra, let $(\bullet, \bullet)$ be a nondegenerate invariant bilinear form on $\mathfrak{g}$, and let $\hat{\mathfrak{g}}$ be the affinization of $\mathfrak{g}$ as constructed in \Cref{exa:4}.
  We now construct $\hat{\mathfrak{g}}$-valued formal distributions by setting
  \begin{equation*}
    a(z) = \sum_{n \in \mathbb{Z}}at^nz^{-n - 1} \quad \text{for $a \in \mathfrak{g}$}.
  \end{equation*}
  By \cite[\S2]{frenkel_vertex_2001}, $\{a(z) \mid a \in \mathfrak{g}\} \cup \{K\}$ is a local family.
  Therefore, $(\hat{\mathfrak{g}}, \{a(z) \mid a \in \mathfrak{g}\} \cup \{K\})$ is a formal distribution Lie algebra.
  Moreover, we can verify directly that $(\hat{\mathfrak{g}}, \{a(z) \mid a \in \mathfrak{g}\} \cup \{K\}, -\partial_t)$ is regular.
\end{example}

Let $V$ be a vector space, and let $a(z) \in \End(V)[[z^{\pm1}]]$ be a formal distribution.
We set
\begin{align*}
  a(z)_+ &= \sum_{n \le -1}a_{(n)}z^{-n - 1}, \\
  a(z)_- &= \sum_{n \ge 0}a_{(n)}z^{-n - 1}.
\end{align*}

A formal distribution $a(z)$ is a \emph{field} if
\begin{equation*}
  a(z)b = \sum_{n \in \mathbb{Z}}a_{(n)}bz^{-n - 1} \in V((z)) \quad \text{for $b \in V$}.
\end{equation*}
The vector space of fields over $V$ is denoted by $\mathcal{F}(V)$.
We note that
\begin{equation*}
  \mathcal{F}(V) = \Hom(V, V((z))).
\end{equation*}
Therefore, we can define a field $a(z)$ by defining $a(z)b \in V((z))$ for $b \in V$.

\begin{proposition}[{\cite[Proposition 3.3.2]{nozaradan_introduction_2008}}]
  \label{prp:2}
  Let $a(z), b(z) \in \mathcal{F}(V)$ be two fields.
  Then $:a(z)b(z): \in \End(V)[[z,z^{-1}]]$ is again a field, where $:a(z)b(z):$ is defined by
  \begin{equation*}
    :a(z)b(z):c = a(z)_+b(z)c + b(z)a(z)_-c \quad \text{for $c \in V$}.
  \end{equation*}
\end{proposition}

We thus defined the notion of \emph{normal ordered product between fields $a(z), b(z) \in \mathcal{F}(V)$}, denoted by $:a(z)b(z):$.
In general, the operation of normal ordered product is neither commutative nor associative.
We follow the convention that the normal ordered product is read from right to left, so that, by definition,
\begin{equation*}
  :a(z)b(z)c(z): = :a(z)(:b(z)c(z):):.
\end{equation*}
We define the normal ordered product of a single field as the field itself, and the normal ordered product of no fields as the identity field $\Id_V$, so we have:
\begin{align*}
  :a(z): &= a(z), \\
  :: &= \Id_V.
\end{align*}
The identity field $\Id_V$ acts as an identity for the normal ordered product, i.e.,
\begin{equation*}
  :\Id_Va(z): = :a(z)\Id_V: = a(z).
\end{equation*}

\begin{lemma}[{\cite[Proposition 3.3.3]{nozaradan_introduction_2008}}]
  \label{lmm:1}
  Let $a(z), b(z) \in \mathcal{F}(V)$ be two fields.
  Their normal ordered product is written explicitly as
  \begin{equation*}
    :a(z)b(z): = \sum_{j \in \mathbb{Z}}:a(z)b(z):_{(j)}z^{-j - 1},
  \end{equation*}
  with
  \begin{equation*}
    :a(z)b(z):_{(j)}c = \sum_{n \le -1}a_{(n)}b_{(j - n - 1)}c + \sum_{n \ge 0}b_{(j - n - 1)}a_{(n)}c \quad \text{for $c \in V$}.
  \end{equation*}
\end{lemma}

\begin{lemma}
  \label{lmm:2}
  Let $V$ be a vector space.
  We consider $s$ fields $a^1(z), \dots, a^s(z) \in \mathcal{F}(V)$, with $s \ge 2$, and let $b \in V$.
  For $l \in \mathbb{Z}$,
  \begin{equation*}
    :a^1(z)a^2(z)\dots a^s(z):_{(l)}b = \sum_{n_1, \dots, n_{s - 1} \in \mathbb{N}}\sum_{k = 0}^{s - 1}R^{l, k}_{n_1, \dots, n_{s - 1}}(a^1(z), \dots, a^s(z))b,
  \end{equation*}
  where $R^{l, k}_{n_1, \dots, n_{s - 1}}(a^1(z), \dots, a^s(z))$ is the sum of $\binom{s - 1}{k}$ terms given by
  \begin{align*}
    &R^{l, k}_{n_1, \dots, n_{s - 1}}(a^1(z), \dots, a^s(z)) = \\
    &\sum_{\substack{1 \le i_1 < \dots < i_k \le s - 1 \\ 1 \le j_1 < \dots < j_{s - 1 - k} \le s - 1 \\ \{i_1, \dots, i_k\} \cup \{j_1, \dots, j_{s - 1 - k}\} = \{1, \dots, s - 1\}}}a^{j_1}_{(-n_{j_1} - 1)}\dots a^{j_{s - 1 - k}}_{(-n_{j_{s - 1 - k}} - 1)}a^s_{(l - k - \sum_{r = 1}^kn_{i_r} + \sum_{r = 1}^{s - 1 - k}n_{j_r})}a^{i_k}_{(n_{i_k})}\dots a^{i_1}_{(n_{i_1})}.
  \end{align*}
\end{lemma}

\begin{proof}
  This follows from \Cref{lmm:3} and induction on $s$.
\end{proof}

We now extend the $n$-products.
Let $a(z), b(z) \in \mathcal{F}(V)$ be two fields.
For $n \in \mathbb{N}$, $\partial^n_za(z)$ is also a field, and we define
\begin{equation}
  \label{eq:3}
  a(z)_{(-n - 1)}b(z) = \frac{:(\partial^n_za(z))b(z):}{n!}.
\end{equation}

A \emph{vertex algebra} is the data consisting of four elements $(V, \vac, T, Y)$ satisfying the following properties:
\begin{enumerate}
\item $V$ is a vector space called the \emph{state space};
\item $\vac \in V$ is called the \emph{vacuum vector};
\item $T \in \End(V)$ is called the \emph{translation operator};
\item $Y: V \to \mathcal{F}(V)$ is a linear map called the \emph{state-field correspondence}, which is commonly written as $Y(a, z) = \sum_{n \in \mathbb{Z}}a_{(n)}z^{-n - 1}$ for $a \in V$.
\end{enumerate}
The data must satisfy the following axioms for $a \in V$:
\begin{enumerate}
\item (Vacuum axiom)
  \begin{align*}
    Y(\vac,z) &= \Id_V, \\
    Y(a, z)\vac &\in V[[z]], \\
    Y(a, z)\vac|_{z = 0} &= a, \\
    T\vac &= 0;
  \end{align*}
\item (Translation covariance) $[T, Y(a, z)] = \partial_zY(a, z)$;
\item (Locality) $\{Y(b, z) \mid b \in V\}$ is a local family of fields.
\end{enumerate}

Formal distribution Lie algebras give us a way to construct nontrivial vertex algebras.
But first, we need a preliminary concept.
A \emph{pre-vertex algebra} is a quadruple $(V, \vac, T, \mathcal{F})$, where $V$ is a vector space, $\vac \in V$, $T \in \End(V)$, and $\mathcal{F} = \{a^j(z) = \sum_{n \in \mathbb{Z}}a^j_{(n)}z^{-n - 1}\}_{j \in J}$ is a collection of $\End(V)$-valued fields.
The above data satisfies the following axioms:
\begin{enumerate}
\item (Vacuum axiom) $T\vac = 0$;
\item (Translation covariance) $[T, a^j(z)] = \partial_za^j(z)$ for $j \in J$;
\item (Locality) $a^i(z)$ and $a^j(z)$ are mutually local for $i, j \in J$;
\item (Completeness) $\vspan\{a^{j_1}_{(n_1)}\dots a^{j_s}_{(n_s)}\vac \mid s \in \mathbb{N}, j_i \in J, n_i \in \mathbb{Z}\} = V$.
\end{enumerate}

A \emph{vertex algebra homomorphism} $f: (V_1, \vac_1, T_1, Y_1) \to (V_2, \vac_2, T_2, Y_2)$ is a linear map $f: V_1 \to V_2$ such that $f(\vac_1) = \vac_2$ and for $a, b \in V_1$,
\begin{equation*}
  f(Y_1(a, z)b) = \sum_{n \in \mathbb{Z}}f(a_{(n)}b)z^{-n - 1} = \sum_{n \in \mathbb{Z}}f(a)_{(n)}f(b)z^{-n - 1} = Y_2(f(a), z)f(b).
\end{equation*}

Let $V$ be a vertex algebra.
A \emph{vertex subalgebra of $V$} is a subspace of $W$ of $V$, which contains $\vac$, and such that $Y(a, z)b \in W((z))$ for $a, b \in W$.
An \emph{ideal of $V$} is a subspace $I$ of $V$ such that $Y(a, z)b \in I((z))$ and $Y(b, z)a \in I((z))$ for $a \in V$ and $b \in I$.
The \emph{quotient vertex algebra} $A/I$ is defined in the usual way.
For example, the kernel of a vertex algebra homomorphism is an ideal.
A vertex algebra is \emph{simple} or \emph{irreducible} if $0$ is the only proper ideal.

The vertex algebra $V$ is \emph{strongly generated by $S \subseteq V$} if
\begin{equation*}
  V = \vspan\{a^1_{(-n_1 - 1)}\dots a^s_{(-n_s - 1)}\vac \mid s, n_1, \dots, n_s \in \mathbb{N}, a^1, \dots, a^s \in S\}.
\end{equation*}

Let $(V, \vac, T, \mathcal{F})$ be a pre-vertex algebra, and we set
\begin{equation*}
  \mathcal{F}_{\max} = \{a(z) \in \mathcal{F}(V) \mid \text{$[T, a(z)] = \partial_za(z)$ and for $j \in J$, $(a(z), a^j(z))$ is a local pair}\}.
\end{equation*}

\begin{theorem}[Extension theorem {\cite[Theorem 1]{callegaro_introduction_2017-1}}]
  \label{thr:4}
  Let $(V, \vac, T, \mathcal{F})$ be a pre-vertex algebra, and let $\mathcal{F}_{\max}$ be defined as above.
  Then:
  \begin{enumerate}
  \item The linear map
    \begin{align*}
      \fs: \mathcal{F}_{\max} &\to V, \\
      \fs(a(z)) &= a(z)\vac|_{z = 0}
    \end{align*}
    is well-defined and bijective, and we denote by $Y: V \to \mathcal{F}(V)$ the inverse map;
  \item $(V, \vac, T, Y)$ is a vertex algebra, with $Y: V \to \mathcal{F}(V)$ given explicitly by
    \begin{equation*}
      Y(a^{j_1}_{(n_1)}a^{j_2}_{(n_2)}\dots a^{j_s}_{(n_s)}\vac) = a^{j_1}(z)_{(n_1)}(a^{j_2}(z)_{(n_2)}\dots(a^{j_s}(z)_{(n_s)}\Id_V)\dots)
    \end{equation*}
    for $s \in \mathbb{N}$, $j_1, \dots, j_s \in J$ and $n_1, \dots, n_s \in \mathbb{Z}$.
  \end{enumerate}
\end{theorem}

Let $(\mathfrak{g}, \mathfrak{F}, T_0)$ be a regular formal distribution Lie algebra with $\mathfrak{F} = \{a^j(z)\}_{j \in J}$, and let $\mathfrak{g}_-$ be the annihilation subalgebra.
Since $T_0(\mathfrak{g}_-) \subseteq \mathfrak{g}_-$, we can extend $T_0$ to a derivation (still denoted the same) $T_0: U(\mathfrak{g}) \to U(\mathfrak{g})$ which is a $(U(\mathfrak{g}), U(\mathfrak{g}_-))$-bimodule homomorphism.
We consider the trivial representation $0: \mathfrak{g}_- \to \mathfrak{gl}(\mathbb{C})$, and we define:
\begin{align*}
  V &= \Ind^{\mathfrak{g}}_{\mathfrak{g}_-}(\mathbb{C}) = U(\mathfrak{g}) \otimes_{U(\mathfrak{g}_-)} \mathbb{C}, \\
  \pi &= \Ind^{\mathfrak{g}}_{\mathfrak{g}_-}(0): \mathfrak{g} \to \mathfrak{gl}(V), \\
  \vac &= 1\otimes1 \in V, \\
  T &= T_0\otimes\Id_{\mathbb{C}} \in \End(V), \\
  \mathcal{F} &= \left\{\pi(a^j(z)) = \sum_{n \in \mathbb{Z}}\pi(a^j_{(n)})z^{-n - 1} \mid j \in J\right\}.
\end{align*}

\begin{theorem}[{\cite[Theorem 3]{callegaro_introduction_2017-1} or \cite[\S1.7]{de_sole_finite_2006}}]
  \label{thr:5}
  With the notation above, $\mathcal{F}$ consists of fields, and $(V, \vac, T, \mathcal{F})$ is a pre-vertex algebra.
\end{theorem}

By \Cref{thr:5} and the Extension theorem, $V$ is a vertex algebra, denoted by $V(\mathfrak{g}, \mathfrak{F}, T_0)$, and by the PBW theorem, it is explicitly given by
\begin{equation*}
  V = \vspan\{a^{j_1}_{(-n_1 - 1)}\dots a^{j_s}_{(-n_s - 1)}\vac \mid s, n_1, \dots, n_s \in \mathbb{N}, j_1, \dots, j_s \in J\}.
\end{equation*}

Usually, we need to quotient the vertex algebras obtained this way.
Let $(\mathfrak{g}, \mathfrak{F}, T_0)$ be a regular formal distribution Lie algebra, and let $\lambda: \mathfrak{h} \to \mathbb{C}$ be a linear functional, where $\mathfrak{h}$ is a subalgebra of $\mathfrak{g}_+$ with $\mathfrak{h} \subseteq \ker(T_0)$.
We denote by $I^{\lambda}$ the $\mathfrak{g}$-submodule of $V(\mathfrak{g}, \mathfrak{F}, T_0)$ generated by the vectors $(a - \lambda(a))\vac$ for $a \in \mathfrak{h}$.
The submodule $I^{\lambda}$ is $T$-invariant and is a left ideal.
Thus, $I^{\lambda}$ is an ideal of $V(\mathfrak{g}, \mathfrak{F}, T_0)$.
Taking the quotient, we get a vertex algebra, denoted by
\begin{equation*}
  V^{\lambda}(\mathfrak{g}, \mathfrak{F}, T_0) = V(\mathfrak{g}, \mathfrak{F}, T_0)/I^{\lambda}.
\end{equation*}

\begin{example}[\emph{Universal Virasoro vertex algebra of central charge $c$}]
  \label{exa:11}
  We pick $c \in \mathbb{C}$.
  We take $(\Vir, \{L(z), C\}, \ad(L_{-1}))$ as the regular formal distribution Lie algebra as constructed in \Cref{exa:9} and $\lambda: \mathbb{C}C \to \mathbb{C}, \lambda(C) = c$ as the linear functional.
  The resulting vertex algebra is the universal Virasoro vertex algebra of central charge $c$, denoted by $\Vir^c$.

  By the PBW theorem, for $c \in \mathbb{C}$, a basis of $\Vir^c$ is given by
  \begin{equation*}
    \{L_{\lambda}\vac \mid \text{$\lambda \vdash$ and ($\lambda = \emptyset$ or $\lambda_{\len(\lambda)} \ge 2$)}\}.
  \end{equation*}
\end{example}

\begin{example}[\emph{Universal affine vertex algebra of level $k$}]
  \label{exa:12}
  We pick $k \in \mathbb{C}$.
  We take $(\hat{\mathfrak{g}}, \{a(z) \mid a \in \mathfrak{g}\} \cup \{K\}, -\partial_t)$ as the regular formal distribution Lie algebra as constructed in \Cref{exa:10} and $\lambda: \mathbb{C}K \to \mathbb{C}, \lambda(K) = k$ as the linear functional.
  The resulting vertex algebra is the universal affine vertex algebra of level $k$, denoted by $V^k(\mathfrak{g})$.
\end{example}

Let $V$ be a vertex algebra.
A \emph{Hamiltonian operator of $V$} is a diagonalizable operator $H \in \End(V)$ such that
\begin{equation}
  \label{eq:4}
  [H, Y(a, z)] = z\partial_zY(a, z) + Y(Ha, z) \quad \text{for $a \in V$}.
\end{equation}
A vertex algebra with a Hamiltonian is called \emph{graded}.
The \emph{grading of $V$} is the eigenspace decomposition of $H$
\begin{equation*}
  V = \bigoplus_{\Delta \in \mathbb{C}}V_{\Delta},
\end{equation*}
where
\begin{equation*}
  V_{\Delta} = \ker(H - \Delta\Id_V) \quad \text{for $\Delta \in \mathbb{C}$}.
\end{equation*}
If $a$ is an eigenvector of $H$, it is called \emph{homogeneous}, its eigenvalue is called the \emph{conformal weight of $a$}, and it is denoted by $\Delta_a$.
Condition \eqref{eq:4} is equivalent to
\begin{equation*}
  [H, a_{(n)}] = -(n + 1)a_{(n)} + (Ha)_{(n)} \quad \text{for $a \in V$ and $n \in \mathbb{Z}$}.
\end{equation*}

\emph{Homomorphisms of graded vertex algebras} are assumed to respect the gradings, i.e., if $f: V_1 \to V_2$ is a homomorphism of graded vertex algebras, then $f\circ H^{V_1} = H^{V_2}\circ f$, where $H^{V_1}$ is the Hamiltonian of $V_1$, and $H^{V_2}$ is the Hamiltonian of $V_2$.

A \emph{conformal vector of central charge $c \in \mathbb{C}$ of $V$} is a vector $\omega \in V$ such that $Y(\omega, z) = \sum_{n \in \mathbb{Z}}L_nz^{-n - 2}$ satisfies:
\begin{enumerate}
\item $Y(\omega, z)$ is a Virasoro formal distribution of central charge $C = c\Id_V$, i.e., it satisfies the Virasoro relations \eqref{eq:1} with $C = c\Id_V$;
\item $L_{-1} = T$;
\item $L_0$ is diagonalizable.
\end{enumerate}

\begin{theorem}[{\cite[\S3.3]{callegaro_introduction_2017-1}}]
  \label{thr:6}
  If $\omega$ is a conformal vector of a vertex algebra $V$, then $L_0$ is a Hamiltonian of $V$, and $\omega$ has conformal weight $2$.
\end{theorem}

\begin{lemma}
  \label{lmm:3}
  In a conformal vertex algebra $V$ such that $V_0 = \mathbb{C}\vac$, there is a unique maximal proper ideal $I_{\max}$ and a unique simple quotient $V/I_{\max}$.
\end{lemma}

\begin{proof}
  We simply let $I_{\max}$ be the sum of all proper ideals.
\end{proof}

A \emph{vertex operator algebra} is a $\mathbb{Z}$-graded conformal vertex algebra such that:
\begin{enumerate}
\item For $n \in \mathbb{Z}$, $\dim(V_n) < \infty$;
\item There is $N \in \mathbb{Z}$ satisfying $V_n = 0$ for $n \le N$.
\end{enumerate}

The conformal vertex algebra $\Vir^c$ satisfies $\Vir^c_0 = \mathbb{C}\vac$.
We denote by $\Vir_c$ the unique simple quotient, and we call it the \emph{simple Virasoro vertex algebra of central charge $c$}.
Let $p, q \ge 2$ be relatively prime integers, and we set
\begin{equation*}
  c_{p, q} = 1 - \frac{6(p - q)^2}{pq}.
\end{equation*}

Let $V$ be a $\Vir$-module.
A vector $u$ in $V$ is called \emph{singular} if it is nonzero and
\begin{equation*}
  L_nu = 0 \quad \text{for $n \in \mathbb{Z}_+$}.
\end{equation*}

\begin{theorem}[{\cite{gorelik_simplicity_2007}}]
  \label{thr:7}
  The following are equivalent:
  \begin{enumerate}
  \item $\Vir^c$ is not simple, i.e., $\Vir^c \neq \Vir_c$;
  \item $c$ is of the form $c_{p, q}$ for some $p, q \ge 2$ relatively prime integers.
  \end{enumerate}
  Moreover, the maximal proper ideal of $\Vir^{c_{p, q}}$ is generated by a singular vector of conformal weight $(p - 1)(q - 1)$, denoted by $a_{p, q}$.
  In the expression
  \begin{equation*}
    a_{p, q} = \sum_{\substack{\lambda \vdash (p - 1)(q - 1) \\ \lambda_{\len(\lambda)} \ge 2}}c_{\lambda}L_{\lambda}\vac,
  \end{equation*}
  where $c_{\lambda} \in \mathbb{Q}$, the coefficient of $L_{-2}^{(p - 1)(q - 1)/2}$ is nonzero.
\end{theorem}

Let $V$ be a vertex algebra.
A \emph{module over $V$}, $V$-module or \emph{representation of $V$} is a vector space $M$ together with a linear map
\begin{align*}
  Y^M(\bullet, z): V &\to \mathcal{F}(M), \\
  a &\mapsto Y^M(a, z) = \sum_{n \in \mathbb{Z}}a^M_{(n)}z^{-n - 1}
\end{align*}
satisfying:
\begin{enumerate}
\item $Y(\vac, z) = \Id_M$;
\item (Borcherds identity) For $a, b \in V$, $u \in M$ and $m, n, k \in \mathbb{Z}$,
  \begin{equation*}
    \sum_{j \in \mathbb{N}}(-1)^j\binom{n}{j}\left(a^M_{(m + n - j)}(b^M_{(k + j)}u) - (-1)^nb^M_{(n + k - j)}(a^M_{(m + j)}u)\right) = \sum_{j \in \mathbb{N}}\binom{m}{j}(a_{(n + j)}b)^M_{(m + k - j)}u.
  \end{equation*}
\end{enumerate}

Let $V$ be a vertex algebra, and let $M_1$, $M_2$ be $V$-modules.
A $V$-module homomorphism $f: M_1 \to M_2$ is a linear map such that for $a \in V$ and $u \in M_1$,
\begin{equation*}
  f(Y^{M_1}(a, z)u) = \sum_{n \in \mathbb{Z}}f(a^{M_1}_{(n)}u)z^{-n - 1} = \sum_{n \in \mathbb{Z}}a^{M_2}_{(n)}f(u)z^{-n - 1} = Y^{M_2}(a, z)f(u).
\end{equation*}

The vertex algebra $V$ is a $V$-module, and it is called the \emph{adjoint representation of $V$}.
A \emph{submodule of $M$} is a subspace $N$ of $M$ such that $Y^M(a, z)u \in N((z))$ for $a \in V$ and $u \in N$, i.e., $a^M_{(n)}u \in N$ for $n \in \mathbb{Z}$.
The \emph{quotient module} $M/N$ is defined in the usual way.
A module whose only proper submodule is $0$ is called \emph{simple} or \emph{irreducible}.

We say \emph{$M$ is strongly generated over $V$ by $T \subseteq M$} if
\begin{equation*}
  M = \vspan\{a^{1M}_{(-n_1 - 1)}\dots a^{sM}_{(-n_s - 1)}u \mid s, n_1, \dots, n_s \in \mathbb{N}, a^1, \dots, a^s \in V, u \in T\}.
\end{equation*}

\begin{proposition}
  \label{prp:3}
  Let $V$ be a vertex algebra, and let $Y^M: V \to \mathcal{F}(M)$ be a $V$-module.
  For $s, n_1, \dots, n_s \in \mathbb{N}$ and $a^1, \dots, a^s \in V$,
  \begin{equation*}
    Y^M(a^1_{(-n_1 - 1)}\dots a^s_{(-n_s - 1)}\vac, z) = \frac{:\partial^{n_1}_zY^M(a^1, z)\dots\partial^{n_s}_zY^M(a^s, z):}{n_1!\dots n_s!}.
  \end{equation*}
\end{proposition}

\begin{proof}
  This is a consequence of the $n$-product identity for modules over vertex algebras (\cite[(5.2.16)]{lepowsky_introduction_2004}) and definition \eqref{eq:3}.
\end{proof}

Let $M$ be a module over a graded vertex algebra $V$ with Hamiltonian $H$.
A \emph{Hamiltonian operator of $M$} is a diagonalizable operator $H^M \in \End(M)$ such that
\begin{equation*}
  [H^M, Y^M(a, z)] = z\partial_zY^M(a, z) + Y^M(Ha, z) \quad \text{for $a \in V$}.
\end{equation*}
A \emph{homomorphism of graded modules} over a graded vertex algebra is assumed to respect the gradings.

Let $(\mathfrak{g}, \mathfrak{F}, T_0)$ be a regular formal distribution Lie algebra with $\mathfrak{F} = \{a^j(z)\}_{j \in J}$.
A \emph{smooth} $(\mathfrak{g}, \mathfrak{F}, T_0)$-module is a $\mathfrak{g}$-module $M$ such for $j \in J$ and $u \in M$, $a^j(z)u \in M((z))$.
Let $\lambda: \mathfrak{h} \to \mathbb{C}$ be a linear functional, where $\mathfrak{h}$ is a subalgebra of $\mathfrak{g}_+$ with $\mathfrak{h} \subseteq \ker(T_0)$.
We say $\mathfrak{h}$ \emph{acts as $\lambda$} if for $h \in \mathfrak{h}$ and $u \in M$, $hu = \lambda(h)u$.

\begin{theorem}[{\cite[Theorem 2.15]{li_vertex_2004}}]
  \label{thr:8}
  Let $(\mathfrak{g}, \mathfrak{F}, T_0)$ be a regular formal distribution Lie algebra with $\mathfrak{F} = \{a^j(z)\}_{j \in J}$.
  Given a smooth $(\mathfrak{g}, \mathfrak{F}, T_0)$-module $M$, there is a unique module structure $Y^M: V(\mathfrak{g}, \mathfrak{F}, T_0) \to \mathcal{F}(M)$ such that
  \begin{equation*}
    Y^M(a^j_{(-1)}\vac, z) = a^j(z) \quad \text{for $j \in J$}.
  \end{equation*}
  Let $\lambda: \mathfrak{h} \to \mathbb{C}$ be a linear functional, where $\mathfrak{h}$ is a subalgebra of $\mathfrak{g}_+$ with $\mathfrak{h} \subseteq \ker(T_0)$.
  Given a smooth $(\mathfrak{g}, \mathfrak{F}, T_0)$-module $M$ where $\mathfrak{h}$ acts as $\lambda$, there is a unique module structure $Y^M_{\lambda}: V^{\lambda}(\mathfrak{g}, \mathfrak{F}, T_0) \to \mathcal{F}(M)$ such that $Y^M_{\lambda}$ factors through $Y^M$, i.e., such that the following diagram commutes
  \begin{equation*}
    \begin{tikzcd}
      V(\mathfrak{g}, \mathfrak{F}, T_0) \arrow[r, two heads] \arrow[rd, "{Y^M}"'] & {V^{\lambda}(\mathfrak{g}, \mathfrak{F}, T_0)} \arrow[d, "{Y^M_{\lambda}}"] \\
      & {\mathcal{F}(M)}
    \end{tikzcd}
  \end{equation*}
  where the horizontal arrow is the quotient map.
\end{theorem}

\begin{example}
  \label{exa:13}
  Let $M$ be a highest weight representation of $\Vir$ with highest weight $(c, h)$.
  Then $M$ is a smooth $\Vir$-module of central charge $c$.
  We note that $L_0$ is diagonalizable.
  By \Cref{thr:8}, $M$ is an $h + \mathbb{N}$-graded $\Vir^c$-module.
\end{example}

We are interested in the irreducible modules over the simple Virasoro vertex algebras $\Vir_{p, q} = \Vir_{c_{p, q}} = L(c_{p, q}, 0)$, where $p, q \ge 2$ are relatively prime integers.
In this context, the following constants play a crucial role.
For integers $m, n$ such that $0 < m < p$ and $0 < n < q$, we set
\begin{equation*}
  h_{m, n} = \frac{(np - mq)^2 - (p - q)^2}{4pq}.
\end{equation*}

\begin{theorem}[{\cite{wang_rationality_1993}}]
  \label{thr:9}
  We set $c = c_{p, q}$ for some $p, q \ge 2$ relatively prime integers.
  Then the irreducible modules over $\Vir_c$ are $L(c, h_{m, n})$ for integers $m, n$ such that $0 < m < p$ and $0 < n < q$.
  Let $Y^{L(c, h_{m, n})}_{\Vir^c}: \Vir^c \to \mathcal{F}(L(c, h_{m, n}))$ be the state-field correspondence of $L(c, h_{m, n})$ as a module over $\Vir^c$, and let $Y^{L(c, h_{m, n})}_{\Vir_c}: \Vir_c \to \mathcal{F}(L(c, h_{m, n}))$ be the state-field correspondence of $L(c, h_{m, n})$ as a module over $\Vir_c$.
  Then $Y^{L(c, h_{m, n})}_{\Vir^c}$ factors through $Y^{L(c, h_{m, n})}_{\Vir_c}$, i.e., the following diagram commutes
  \begin{equation*}
    \begin{tikzcd}
      \Vir^c \arrow[r, two heads] \arrow[rd, "{Y^{L(c, h_{m, n})}_{\Vir^c}}"'] & {\Vir_c} \arrow[d, "{Y^{L(c, h_{m, n})}_{\Vir_c}}"] \\
      & {\mathcal{F}(L(c, h_{m, n}))}
    \end{tikzcd}
  \end{equation*}
  where the horizontal arrow is the quotient map.
\end{theorem}

We are also interested in the refined character of these irreducible modules.
The ordinary character is fortunately already known.

\begin{theorem}[{\cite{feigin_verma_1984}}]
  \label{thr:10}
  Let $p, q \ge 2$ be relatively prime integers, and let $m, n$ be integers such that $0 < m < p$ and $0 < n < q$.
  Then
  \begin{align*}
    \ch_{L(c_{p, q}, h_{m, n})}(q) &= \frac{1}{(q)_{\infty}}\sum_{k \in \mathbb{Z}}q^{\frac{(2kpq + mq - np)^2 - (p - q)^2}{4pq}} - q^{\frac{(2kpq + mq + np)^2 - (p - q)^2}{4pq}} \\
                                   &= \frac{q^{h_{m, n}}}{(q)_{\infty}}\sum_{k \in \mathbb{Z}}q^{k^2pq + k(mq - np)} - q^{k^2pq + k(mq + np) + mn}.
  \end{align*}
\end{theorem}

In this article, we consider:
\begin{align*}
  p &= 2, \\
  q &= 2s + 1 \quad \text{for $s \in \mathbb{Z}_+$}.
\end{align*}
By \Cref{thr:9}, $\Vir_{2, 2s + 1}$ has $s$ irreducible modules $L(c_{2, 2s + 1}, h_{1, 1}), \dots, L(c_{2, 2s + 1}, h_{1, s})$, known collectively as the \emph{boundary minimal models}.

\section{Graded vertex Poisson algebras and their modules}
\label{sec:graded-vert-poiss}

Let $V$ be a vector space.
Given a formal distribution $f(x_1, \dots, x_n) \in V[[x_1^{\pm1}, \dots, x_n^{\pm1}]]$, we can write it as
\begin{equation*}
  f(x_1, \dots, x_n) = \sum_{m_1, \dots, m_n \in \mathbb{Z}}f_{(m_1, \dots, m_n)}x_1^{-m_1 - 1}\dots x_n^{-m_n - 1}.
\end{equation*}
We set
\begin{equation*}
  \sing(f(x_1, \dots, x_n)) = \sum_{m_1, \dots, m_n \in \mathbb{N}}f_{(m_1, \dots, m_n)}x_1^{-m_1 - 1}\dots x_n^{-m_n - 1}.
\end{equation*}

A \emph{vertex Lie algebra} is the data $(V, T, Y_-)$, where $V$ is a vector space, $Y_-: V \to \mathcal{F}(V)$ is a linear map such that $Y_-(a, z) = \sing(Y_-(a, z))$ (i.e., $Y_-: V \to \Hom(V, z^{-1}V[z^{-1}])$) for $a \in V$ and $T \in \End(V)$.
The data must satisfy the following axioms for $a, b \in V$:
\begin{enumerate}
\item $Y_-(Ta, z) = \partial_zY_-(a, z)$;
\item $Y_-(a, z)b = \sing(e^{Tz}Y_-(b, -z)a)$;
\item $[Y_-(a, z), Y_-(b, w)] = \sing(\sum_{j \in \mathbb{N}}\frac{\partial^j_w\delta(z, w)}{j!}Y_-(a_{(j)}b, w))$, where $Y_-(a, z) = \sum_{n \in \mathbb{N}}a_{(n)}z^{-n - 1}$, $a_{(n)} \in \End(V)$ and
  \begin{equation*}
    \delta(z, w) = \sum_{n \in \mathbb{Z}}z^nw^{-n - 1} \in \mathbb{C}[[z^{\pm1}, w^{\pm1}]]
  \end{equation*}
  is the \emph{formal delta distribution}.
\end{enumerate}

Expanding the definitions, axiom (iii) of vertex Lie algebras is equivalent to
\begin{equation*}
  [a_{(m)}, b_{(n)}] = \sum_{j = 0}^m\binom{m}{j}(a_{(j)}b)_{(m + n - j)} \quad \text{for $a, b \in V$ and $m, n \in \mathbb{N}$}.
\end{equation*}

\begin{theorem}[{\cite{li_vertex_2004}}]
  \label{thr:11}
  Let $V$ be a vertex Lie algebra.
  For $a, b \in V$ and $m, j \in \mathbb{N}$:
  \begin{enumerate}
  \item $[T, Y_-(a, z)] = Y_-(Ta, z) = \partial_zY_-(a, z)$;
  \item $(a_{(j)}b)_{(m)} = \sum_{k = 0}^j\binom{j}{k}(-1)^k[a_{(j - k)}, b_{(m + k)}]$.
  \end{enumerate}
\end{theorem}

Let $V$ be a vertex Lie algebra.
A \emph{Hamiltonian operator of $V$} is a diagonalizable operator $H \in \End(V)$ such that:
\begin{enumerate}
\item $[H, T] = T$;
\item $[H, Y_-(a, z)] = z\partial_zY_-(a, z) + Y_-(Ha, z)$ for $a \in V$.
\end{enumerate}
A \emph{module over $V$} is a vector space $M$ together with a linear map $Y^M_-: V \to \Hom(M, z^{-1}M[z^{-1}])$, written as $Y^M_-(a, z) = \sum_{n \in \mathbb{N}}a^M_{(n)}z^{-n - 1}$, $a^M_{(n)} \in \End(M)$, such that for $a, b \in V$ and $m, n \in \mathbb{N}$:
\begin{enumerate}
\item $(Ta)^M_{(n)} = -na^M_{(n - 1)}$;
\item $[a^M_{(m)}, b^M_{(n)}] = \sum_{j = 0}^m\binom{m}{j}(a_{(j)}b)^M_{(m + n - j)}$.
\end{enumerate}

Let $M$ be a module over a graded vertex Lie algebra $V$ with Hamiltonian $H$.
A \emph{Hamiltonian operator of $M$} is a diagonalizable operator $H^M \in \End(M)$ such that
\begin{equation*}
  [H^M, Y^M_-(a, z)] = z\partial_zY^M_-(a, z) + Y^M_-(Ha, z) \quad \text{for $a \in V$}.
\end{equation*}

A \emph{vertex Poisson algebra} is the data consisting of three elements $(V, T, Y_-)$ such that:
\begin{enumerate}
\item $(V, T)$ is a differential commutative associative algebra with unit $1$;
\item $(V, T, Y_-)$ is a vertex Lie algebra;
\item The left Leibniz rule holds
  \begin{equation*}
    Y_-(a, z)(bc) = (Y_-(a, z)b)c + b(Y_-(a, z)c) \quad \text{for $a, b, c \in V$}.
  \end{equation*}
\end{enumerate}

Let $V$ be a vertex Poisson algebra.
A \emph{Hamiltonian operator of $V$} is simultaneously a Hamiltonian of $V$ as a differential commutative associative algebra with derivation $T$ (see \Cref{sec:grad-poiss-algeb}) and a Hamiltonian of $V$ as a vertex Lie algebra.
A \emph{module over $V$} is a module $(M, Y^M_-)$ over $V$ as a vertex Lie algebra and a module over $V$ as a commutative associative algebra with unit $1$ such that
\begin{equation*}
  a^M_{(n)}(bu) = (a_{(n)}b)u + b(a^M_{(n)}u) \quad \text{for $a, b \in V$, $u \in M$ and $n \in \mathbb{N}$}.
\end{equation*}

Let $M$ be a module over a graded vertex Poisson algebra $V$ with Hamiltonian $H$.
A \emph{Hamiltonian operator of $M$} is simultaneously a Hamiltonian of $M$ as a module over $V$ as a commutative associative algebra with unit $1$ and a Hamiltonian of $M$ as a module over $V$ as a vertex Lie algebra.

\begin{proposition}[{\cite[Proposition 2.3.1]{arakawa_remark_2012}} and {\Cref{sec:graded-jet-algebras}}]
  \label{prp:4}
  Let $R$ be a (graded) Poisson algebra.
  Then there is a unique (graded) vertex Poisson algebra structure on $JR$ such that
  \begin{equation*}
    a_{(n)}b = \delta_{n, 0}\{a, b\} \quad \text{for $a, b \in R$ and $n \in \mathbb{N}$}.
  \end{equation*}
\end{proposition}

The (graded) vertex Poisson algebra structure on $JR$ given in \Cref{prp:4} for a (graded) Poisson algebra $R$ will be called the \emph{level 0 vertex Poisson algebra structure of $JR$}.

If $R$ is a (graded) Poisson algebra, and $M$ is a (graded) $R$-module, then we can verify that $JR \otimes_R M$ is a (graded) $JR$-module by defining the Poisson structure as
\begin{equation*}
  a_{(n)}(b\otimes u) = (a_{(n)}b)\otimes u + \delta_{n, 0}b\otimes\{a, u\} \quad \text{for $a \in R$, $b \in JR$, $u \in M$ and $n \in \mathbb{N}$}
\end{equation*}
and $H^{JR \otimes_R M}$ as in \Cref{sec:graded-jet-algebras}.
We have a natural (graded) inclusion $\inc: M \hookrightarrow JR \otimes_R M$, which is a (graded) $R$-module homomorphism.
The (graded) $JR$-module $M$ together with the (graded) $R$-module inclusion $\inc: M \hookrightarrow JR \otimes_R M$ satisfy a universal property similar to that of $JR$ (see \Cref{sec:graded-jet-algebras}), as we now show.
Let $N$ be a (graded) module over the (graded) vertex Poisson algebra $JR$, and let $f: M \to N$ be a (graded) homomorphism of modules over the (graded) Poisson algebra $R$.
Thus, $N$ is a module over the (graded) algebra $JR$ and in particular, a module over the (graded) algebra $R$ because we have an inclusion $\inc: R \hookrightarrow JR$.
Also, $f$ can be considered as a (graded) algebra homomorphism.
Therefore, there is a unique (graded) homomorphism $\overline{f}: JR \otimes_R M \to N$ of (graded) modules over the (graded) algebra $JR$ such that the following diagram commutes
\begin{equation*}
  \begin{tikzcd}
    M \arrow[r, "\inc", hook] \arrow[rd, "f"'] & JR \otimes_R M \arrow[d, "\overline{f}"] \\
    & N
  \end{tikzcd}
\end{equation*}

\begin{proposition}
  \label{prp:5}
  The homomorphism $\overline{f}$ as defined above is a (graded) $JR$-module homomorphism.
\end{proposition}

\begin{proof}
  We just expand the definitions and verify the axioms of (graded) modules over (graded) vertex Poisson algebras.
\end{proof}

We have constructed a functor
\begin{equation*}
  JR \otimes_R \bullet: \{\text{(graded) $R$-modules}\} \to \{\text{(graded) $JR$-modules}\},
\end{equation*}
which is left adjoint to the forgetful functor $\{\text{(graded) $JR$-modules}\} \to \{\text{(graded) $R$-modules}\}$, i.e., for $M \in \{\text{(graded) $R$-modules}\}$ and $N \in \{\text{(graded) $JR$-modules}\}$, we have a natural isomorphism
\begin{equation*}
  \Hom_{\{\text{(graded) $JR$-modules}\}}(JR \otimes_R M, N) \cong \Hom_{\{\text{(graded) $R$-modules}\}}(M, N).
\end{equation*}

\section{Filtrations of vertex algebras and their modules}
\label{sec:filtr-vert-algebr}

Let $V$ be a vertex algebra, and let $(a^i)_{i \in I}$ be a family of strong generators of $V$.
For $p \in \mathbb{Z}$, we set
\begin{equation*}
  F_pV = \vspan\{a^{i_1}_{(-n_1 - 1)}\dots a^{i_s}_{(-n_s - 1)}\vac \mid s, n_1, \dots, n_s \in \mathbb{N}, i_1, \dots, i_s \in I, n_1 + \dots + n_s \ge p\}.
\end{equation*}

\begin{proposition}[{\cite{li_abelianizing_2005}}]
  \label{prp:6}
  The filtration $(F_pV)_{p \in \mathbb{Z}}$ satisfies:
  \begin{enumerate}
  \item $F_pV = V$ for $p \le 0$;
  \item $\vac \in F_0V \supseteq F_1V \supseteq \dots$;
  \item $T(F_pV) \subseteq F_{p + 1}V$ for $p \in \mathbb{Z}$;
  \item $a_{(n)}F_qV \subseteq F_{p + q - n - 1}V$ for $p, q \in \mathbb{Z}$, $a \in F_pV$ and $n \in \mathbb{Z}$;
  \item $a_{(n)}F_qV \subseteq F_{p + q - n}V$ for $p, q \in \mathbb{Z}$, $a \in F_pV$ and $n \in \mathbb{N}$.
  \end{enumerate}
\end{proposition}

Let
\begin{equation*}
  \gr_F(V) = \bigoplus_{p \in \mathbb{N}}F_pV/F_{p + 1}V
\end{equation*}
be the associated graded vector space.
By \cite{li_abelianizing_2005}, the vector space $\gr_F(V)$ is a vertex Poisson algebra with operations given as follows.
For $p, q \in \mathbb{N}$, $a \in F_pV$ and $b \in F_qV$, we set:
\begin{align*}
  \sigma_p(a)\sigma_q(b) &= \sigma_{p + q}(a_{(-1)}b), \\
  T(\sigma_p(a)) &= \sigma_{p + 1}(Ta), \\
  Y_-(\sigma_p(a), z)\sigma_q(b) &= \sum_{n \in \mathbb{N}}\sigma_{p + q - n}(a_{(n)}b)z^{-n - 1},
\end{align*}
where $\sigma_p: F_pV \to \gr_F(V)$ is the \emph{principal symbol map}, which is the composition of the natural maps $F_pV \twoheadrightarrow F_pV/F_{p + 1}V$ and $F_pV/F_{p + 1}V \hookrightarrow \gr_F(V)$.
The unit is $\sigma^0(\vac)$.
The filtration $(F_pV)_{p \in \mathbb{Z}}$ is called the \emph{Li filtration of $V$}.

\begin{lemma}[{\cite[Lemma 2.9]{li_abelianizing_2005}}]
  \label{lmm:4}
  Let $V$ be a vertex algebra.
  Then
  \begin{equation*}
    F_pV = \vspan\{a_{(-i - 1)}b \mid a \in V, i \in \mathbb{Z}_+, b \in F_{p - i}V\} \quad \text{for $p \in \mathbb{Z}_+$}.
  \end{equation*}
\end{lemma}

By \Cref{lmm:4}, the Li filtration depends only on $V$ and not on the choice of the strong generators.
If $V$ is graded by a Hamiltonian $H$ with grading $V = \bigoplus_{\Delta \in \mathbb{C}}V_{\Delta}$, then $H(F_pV) \subseteq F_pV$ because in that case, for $p \in \mathbb{Z}$,
\begin{equation*}
  \begin{split}
    F_pV = \vspan\{a^1_{(-n_1 - 1)}\dots a^s_{(-n_s - 1)}\vac &\mid \text{$s, n_1, \dots, n_s \in \mathbb{N}$, $a^1, \dots, a^s \in V$ homogeneous,} \\
                                                              &\quad n_1 + \dots + n_s \ge p\}.
  \end{split}
\end{equation*}

Therefore, we can define an operator $H \in \End(\gr_F(V))$ as $H(\sigma_p(a)) = \sigma_p(Ha)$ for $p \in \mathbb{N}$ and $a \in F_pV$.
For $p \in \mathbb{Z}$ and $\Delta \in \mathbb{C}$, we define $F_pV_{\Delta} = F_pV \cap V_{\Delta}$.
Since $H(F_pV) \subseteq F_pV$ for $p \in \mathbb{Z}$, we have
\begin{equation*}
  F_pV = \bigoplus_{\Delta \in \mathbb{C}}F_pV_{\Delta} \quad \text{for $p \in \mathbb{Z}$}.
\end{equation*}
For $\Delta \in \mathbb{C}$, we define $\gr_F(V)_{\Delta} = \bigoplus_{p \in \mathbb{N}}\sigma_p(F_pV_{\Delta})$.
Then $Ha = \Delta a$ for $a \in \gr_F(V)_{\Delta}$.
The family of subspaces $(\gr_F(V)_{\Delta})_{\Delta \in \mathbb{C}}$ satisfies $\gr_F(V) = \bigoplus_{\Delta \in \mathbb{C}}\gr_F(V)_{\Delta}$.
Therefore, the operator $H \in \End(\gr_F(V))$ is diagonalizable with $\gr_F(V)_{\Delta} = \ker(H - \Delta\Id_{\gr_F(V)})$.
In fact, it is possible to prove that this diagonalizable operator $H$ is actually a Hamiltonian of $\gr_F(V)$.

We have the natural vector space isomorphisms
\begin{equation*}
  \sigma_p(F_pV_{\Delta}) \cong F_pV_{\Delta}/F_{p + 1}V_{\Delta} \quad \text{for $p \in \mathbb{Z}$ and $\Delta \in \mathbb{C}$}
\end{equation*}
and the refined grading
\begin{equation}
  \label{eq:5}
  \gr_F(V) = \bigoplus_{\substack{p \in \mathbb{N} \\ \Delta \in \mathbb{C}}}\sigma_p(F_pV_{\Delta}).
\end{equation}
By \eqref{eq:5}, when $\dim(V_{\Delta}) < \infty$ for $\Delta \in \mathbb{C}$, it is natural to define the \emph{refined character of $V$ with respect to the Li filtration} as
\begin{equation*}
  \ch_{\gr_F(V)}(t, q) = \sum_{\substack{p \in \mathbb{N} \\ \Delta \in \mathbb{C}}}\dim(\sigma_p(F_pV_{\Delta}))t^pq^{\Delta}.
\end{equation*}

If $f: V_1 \to V_2$ is a homomorphism of vertex algebras, then
\begin{align*}
  \gr_F(f): \gr_F(V_1) &\to \gr_F(V_2), \\
  \gr_F(f)(\sigma^{V_1}_p(a)) &= \sigma^{V_2}_p(f(a)) \quad \text{for $p \in \mathbb{N}$ and $a \in F_pV_1$}
\end{align*}
defines a homomorphism of vertex Poisson algebras.
If $V_1$ and $V_2$ are graded, then we require that $f$ respects the gradings of $V_1$ and $V_2$, and this implies that $\gr_F(f)$ also respects the gradings of $\gr_F(V_1)$ and $\gr_F(V_2)$.
Therefore, we obtain a functor
\begin{equation*}
  \gr_F: \{\text{(graded) vertex algebras}\} \to \{\text{(graded) vertex Poisson algebras}\}.
\end{equation*}

\begin{example}[$\gr_F(\Vir^c)$]
  \label{exa:14}
  We pick $c \in \mathbb{C}$.
  We have an isomorphism
  \begin{align*}
    \gr_F(\Vir^c) &\xrightarrow{\sim} \mathbb{C}[L_{-2}, L_{-3}, \dots] \\
    \sigma_{\Delta(\lambda) - \clen(\lambda)}(L_{\lambda}\vac) &\mapsto p_{\lambda} \quad \text{for $\lambda$ a composition without ones}.
  \end{align*}
  The endomorphism $T \in \End(\mathbb{C}[L_{-2}, L_{-3}, \dots])$ is given by $T(L_{-n}) = (n - 1)L_{-n - 1}$ for $n \ge 2$, which is extended to a derivation.
  The Poisson structure of $\gr^G(\Vir^c)$ is trivial (i.e., the map $Y_-$ is zero).
  The refined character is given explicitly by
  \begin{equation*}
    \ch_{\gr_F(\Vir^c)}(t, q) = \frac{1}{\prod_{k \ge 2}(1 - t^{k - 2}q^k)}.
  \end{equation*}
\end{example}

Let $V$ be a vertex algebra, let $(a^i)_{i \in I}$ be a family of strong generators of $V$, and let $M$ be a $V$-module.
For $p \in \mathbb{Z}$, we set
\begin{equation*}
  F_pM = \vspan\{a^{i_1M}_{(-n_1 - 1)}\dots a^{i_sM}_{(-n_s - 1)}u \mid s, n_1, \dots, n_s \in \mathbb{N}, i_1, \dots, i_s \in I, u \in M, n_1 + \dots + n_s \ge p\}.
\end{equation*}

\begin{proposition}[{\cite{li_abelianizing_2005}}]
  \label{prp:7}
  The filtration $(F_pM)_{p \in \mathbb{Z}}$ satisfies:
  \begin{enumerate}
  \item $M = F_pM$ for $p \le 0$;
  \item $F_0M \supseteq F_1M \supseteq \dots$;
  \item $a_{(n)}F_qM \subseteq F_{p + q - n - 1}M$ for $p, q \in \mathbb{Z}$, $a \in F_pV$ and $n \in \mathbb{Z}$;
  \item $a_{(n)}F_qM \subseteq F_{p + q - n}M$ for $p, q \in \mathbb{Z}$, $a \in F_pV$ and $n \in \mathbb{N}$.
  \end{enumerate}
\end{proposition}

Let
\begin{equation*}
  \gr_F(M) = \bigoplus_{p \in \mathbb{N}}F_pM/F_{p + 1}M
\end{equation*}
be the associated graded vector space.
By \cite{li_abelianizing_2005}, the vector space $\gr_F(M)$ is a module over $\gr_F(V)$ with operations given as follows.
For $p, q \in \mathbb{N}$, $a \in F_pV$ and $u \in F_pM$, we set:
\begin{align*}
  \sigma_p(a)\sigma^M_q(u) &= \sigma^M_{p + q}(a^M_{(-1)}u), \\
  Y^M_-(\sigma_p(a), z)\sigma^M_q(u) &= \sum_{n \in \mathbb{N}}\sigma^M_{p + q - n}(a^M_{(n)}u)z^{-n - 1},
\end{align*}
where $\sigma^M_p: F_pM \to \gr_F(M)$ is the principal symbol map.
The filtration $(F_pM)_{p \in \mathbb{Z}}$ is called the Li filtration of $M$.

\begin{lemma}[{\cite[Lemma 2.9]{li_abelianizing_2005}}]
  \label{lmm:5}
  Let $V$ be a vertex algebra, and let $M$ be a $V$-module.
  Then
  \begin{equation*}
    F_pM = \vspan\{a^M_{(-i - 1)}u \mid a \in V, i \in \mathbb{Z}_+, u \in F_{p - i}M\} \quad \text{for $p \in \mathbb{Z}_+$}.
  \end{equation*}
\end{lemma}

By \Cref{lmm:5}, the Li filtration depends only on $M$ and not on the choice of the strong generators of $V$.
If $V$ is graded by a Hamiltonian $H$, and $M$ is graded by a Hamiltonian $H^M$, then $H^M(F_pM) \subseteq F_pM$ because in that case, for $p \in \mathbb{Z}$,
\begin{equation*}
  \begin{split}
    F_pM = \vspan\{a^{1M}_{(-n_1 - 1)}\dots a^{sM}_{(-n_s - 1)}u &\mid \text{$s, n_1, \dots, n_s \in \mathbb{N}$, $a^1, \dots, a^s \in V$ homogeneous,} \\
                                                                 &\quad \text{$u \in M$ homogeneous, $n_1 + \dots + n_s \ge p$}\}.
  \end{split}
\end{equation*}

Therefore, we can define an operator $H^M \in \End(\gr_F(M))$ as $H^M(\sigma^M_p(u)) = \sigma^M_p(H^Mu)$ for $p \in \mathbb{N}$ and $u \in F_pM$.
For $p \in \mathbb{Z}$ and $\Delta \in \mathbb{C}$, we define $F_pM_{\Delta} = F_pM \cap M_{\Delta}$.
Since $H^M(F_pM) \subseteq F_pM$ for $p \in \mathbb{Z}$, we have
\begin{equation*}
  F_pM = \bigoplus_{\Delta \in \mathbb{C}}F_pM_{\Delta} \quad \text{for $p \in \mathbb{Z}$}.
\end{equation*}
For $\Delta \in \mathbb{C}$, we define $\gr_F(M)_{\Delta} = \bigoplus_{p \in \mathbb{N}}\sigma^M_p(F_pM_{\Delta})$.
Then $H^Mu = \Delta u$ for $u \in \gr_F(M)_{\Delta}$.
The family of subspaces $(\gr_F(M)_{\Delta})_{\Delta \in \mathbb{C}}$ satisfies $\gr_F(M) = \bigoplus_{\Delta \in \mathbb{C}}\gr_F(M)_{\Delta}$.
Therefore, the operator $H^M \in \End(\gr_F(M))$ is diagonalizable with $\gr_F(M)_{\Delta} = \ker(H^M - \Delta\Id_{\gr_F(M)})$.
In fact, it is possible to prove that this diagonalizable operator $H^M$ is actually a Hamiltonian of $\gr_F(M)$.

We have the natural vector space isomorphisms
\begin{equation*}
  \sigma^M_p(F_pM_{\Delta}) \cong F_pM_{\Delta}/F_{p + 1}M_{\Delta} \quad \text{for $p \in \mathbb{Z}$ and $\Delta \in \mathbb{C}$}
\end{equation*}
and the refined grading
\begin{equation}
  \label{eq:6}
  \gr_F(M) = \bigoplus_{\substack{p \in \mathbb{N} \\ \Delta \in \mathbb{C}}}\sigma_p(F_pM_{\Delta}).
\end{equation}
By \eqref{eq:6}, when $\dim(M)_{\Delta} < \infty$ for $\Delta \in \mathbb{C}$, it is natural to define the refined character of $M$ with respect to the Li filtration as
\begin{equation*}
  \ch_{\gr_F(M)}(t, q) = \sum_{\substack{p \in \mathbb{N} \\ \Delta \in \mathbb{C}}}\dim(\sigma_p(F_pM_{\Delta}))t^pq^{\Delta}.
\end{equation*}

If $f: M_1 \to M_2$ is a homomorphism of $V$-modules, then
\begin{align*}
  \gr_F(f): \gr_F(M_1) &\to \gr_F(M_2), \\
  \gr_F(f)(\sigma^{M_1}_p(u)) &= \sigma^{M_2}_p(f(u)) \quad \text{for $p \in \mathbb{N}$ and $u \in F_pM_1$}
\end{align*}
defines a homomorphism of $\gr_F(V)$-modules.
If $M_1$ and $M_2$ are graded, then we require that $f$ respects the gradings of $M_1$ and $M_2$, and this implies that $\gr_F(f)$ also respects the gradings of $\gr_F(M_1)$ and $\gr_F(M_2)$.
Therefore, we obtain a functor
\begin{equation*}
  \gr_F: \{\text{(graded) $V$-modules}\} \to \{\text{(graded) $\gr_F(V)$-modules}\}.
\end{equation*}

\begin{example}[$\gr_F(M(c, h))$]
  \label{exa:15}
  We pick a highest weight $(c, h)$.
  From \Cref{exa:14}, $\gr_F(\Vir^c)$ is isomorphic to $\mathbb{C}[L_{-2}, L_{-3}, \dots]$.
  We have an isomorphism
  \begin{align*}
    \gr_F(M(c, h)) &\xrightarrow{\sim} \bigoplus_{k \in \mathbb{N}}\mathbb{C}[L_{-2}, L_{-3}, \dots]L_{-1}^k \\
    \sigma_{\Delta(\lambda) - \clen(\lambda)}(L_{\lambda}|c, h\rangle) &\mapsto u_{\lambda} \quad \text{for $\lambda$ a composition}.
  \end{align*}
  The refined character is given explicitly by
  \begin{equation*}
    \ch_{\gr_F(M(c, h))}(t, q) = \frac{q^h}{(1 - q)\prod_{k \ge 2}(1 - t^{k - 2}q^k)}.
  \end{equation*}
\end{example}

\section{Classically free vertex algebras and their modules}
\label{sec:class-free-vert}

Let $V$ be a vertex algebra.
By \Cref{lmm:4}, we have
\begin{equation*}
  C_2V = \vspan\{a_{(-2)}b \mid a, b \in V\} = F_1V.
\end{equation*}
We define the \emph{Zhu $C_2$-algebra of $V$} by
\begin{equation*}
  R_V = V/C_2V = F_0V/F_1V \subseteq \gr_F(V).
\end{equation*}
In Zhu's original article \cite{zhu_modular_1996} and other works, $R_V$ is denoted by $A(V)$.

The fact that $\gr_F(V)$ is a vertex Poisson algebra implies that $R_V$ is a Poisson algebra with operations given as follows.
For $a, b \in V$, we set:
\begin{align*}
  \sigma_0(a)\sigma_0(b) &= \sigma_0(a_{(-1)}b), \\
  \{\sigma_0(a), \sigma_0(b)\} &= \sigma_0(a_{(0)}b).
\end{align*}

If $V$ is graded, then, as we explained in \Cref{sec:filtr-vert-algebr}, $\gr_F(V)$ is graded.
Thus, $R_V$ is also graded.

We have constructed a functor
\begin{equation*}
  R: \{\text{(graded) vertex algebras}\} \to \{\text{(graded) Poisson algebras}\}.
\end{equation*}

Oftentimes, some condition on $R_V$ implies or is equivalent to some condition on $V$, as we shall see in this section.
The vertex algebra $V$ is called \emph{$C_2$-cofinite} if $R_V$ is finite dimensional.

\begin{example}[$R_{\Vir^c}$ and $R_{\Vir_c}$]
  \label{exa:16}
  We pick $c \in \mathbb{C}$.
  We consider $\mathbb{C}[L_{-2}]$ as the polynomial algebra in one variable $L_{-2}$, and we equip it with the trivial Poisson bracket.
  By \Cref{exa:14}, we have the following isomorphism of Poisson algebras
  \begin{align*}
    R_{\Vir^c} &\xrightarrow{\sim} \mathbb{C}[L_{-2}], \\
    \sigma_0(L_{-2}\vac) &\mapsto L_{-2}.
  \end{align*}

  We now move onto $\Vir_c$.
  If $c$ is not of the form $c_{p, q}$ for some $p, q \ge 2$ relatively prime integers, then $\Vir_c = \Vir^c$ by \Cref{thr:7}, and we have already solved the problem.
  Therefore, we assume $c$ is of this form.

  We have a natural quotient map
  \begin{align*}
    \pi_c: \Vir^c &\twoheadrightarrow \Vir_c, \\
    \pi_c(a) &= a + U(\Vir)\{a_{p, q}\}.
  \end{align*}
  Applying the functor $R$, we obtain an epimorphism
  \begin{equation*}
    R_{\pi_c}: R_{\Vir^c} \twoheadrightarrow R_{\Vir_c}.
  \end{equation*}
  From the equation $\ker(R_{\pi_c}) = \sigma_0(U(\Vir)\{a_{p, q}\})$ and \Cref{thr:7}, we obtain
  \begin{equation*}
    \ker(R_{\pi_c}) = (\sigma_0(L_{-2}^{(p - 1)(q - 1)/2}\vac)).
  \end{equation*}
  In summary,
  \begin{equation*}
    R_{\Vir_{p, q}} \cong \mathbb{C}[L_{-2}]/(L_{-2}^{(p - 1)(q - 1)/2}).
  \end{equation*}
  Thus, $\Vir^c$ is never $C_2$-cofinite, while $\Vir_c$ is $C_2$-cofinite only when $c$ is of the form $c_{p, q}$ for some $p, q \ge 2$ relatively prime integers.
\end{example}

\begin{lemma}[{\cite[Corollary 4.3]{li_abelianizing_2005}}]
  \label{lmm:6}
  Let $V$ be a vertex algebra.
  As a differential algebra, $\gr_F(V)$ is generated by $R_V$, i.e.,
  \begin{equation*}
    \gr_F(V) = (R_V)_T.
  \end{equation*}
\end{lemma}

Let $V$ be a (graded) vertex algebra.
We have a natural (graded) algebra inclusion $\inc: R_V \hookrightarrow \gr_F(V)$.
By the universal property of $JR_V$ (see \Cref{sec:graded-jet-algebras}), there is a (graded) differential algebra homomorphism $\phi_V: JR_V \to \gr_F(V)$ such that the following diagram commutes
\begin{equation*}
  \begin{tikzcd}
    R_V \arrow[rd, "\inc"', hook] \arrow[r, "\inc", hook] & JR_V \arrow[d, "\phi_V"] \\
    & \gr_F(V)
  \end{tikzcd}
\end{equation*}
Because $R_V$ is a (graded) Poisson algebra, we can equip $JR_V$ with the level 0 vertex Poisson algebra structure.
From now on, $JR_V$ will be considered as a (graded) vertex Poisson algebra.

\begin{lemma}
  \label{lmm:7}
  Let $V$ be a (graded) vertex algebra.
  The (graded) differential algebra homomorphism $\phi_V: JR_V \to \gr_F(V)$ defined above is surjective and is actually a (graded) vertex Poisson algebra homomorphism.
\end{lemma}

\begin{proof}
  The homomorphism $\phi_V$ is surjective by \Cref{lmm:6}.
  The fact that $\phi_V$ is a (graded) vertex Poisson algebra homomorphism is explained in \cite[Proposition 2.5.1]{arakawa_remark_2012}.
\end{proof}

When $\phi_V$ is an isomorphism, we say $V$ is \emph{classically free}.

\begin{example}[$JR_{\Vir^c}$ and $JR_{\Vir_c}$]
  \label{exa:17}
  Let $c \in \mathbb{C}$.
  Then $\Vir^c$ is classically free because
  \begin{equation*}
    \gr_F(\Vir^c) \cong \mathbb{C}[L_{-2}, L_{-3}, \dots]
  \end{equation*}
  by \Cref{exa:14} and
  \begin{equation*}
    JR_{\Vir^c} \cong J(C[L_{-2}]) = \mathbb{C}[L_{-2}, L_{-3}, \dots]
  \end{equation*}
  by \Cref{exa:16} and \Cref{sec:graded-jet-algebras}.

  We now move onto $\Vir_c$.
  If $c$ is not of the form $c_{p, q}$ for some $p, q \ge 2$ relatively prime integers, then $\Vir_c = \Vir^c$ by \Cref{thr:7}, and we have already solved the problem.
  Therefore, we assume $c$ is of this form.
  Then
  \begin{equation*}
    JR_{\Vir_{p, q}} \cong J(\mathbb{C}[L_{-2}]/(L_{-2}^{(p - 1)(q - 1)/2})) = \mathbb{C}[L_{-2}, L_{-3}, \dots]/(L_{-2}^{(p - 1)(q - 1)/2})_{\partial}
  \end{equation*}
  by \Cref{exa:16} and \Cref{sec:graded-jet-algebras}.
\end{example}

\begin{example}
  \label{exa:18}
  If $p > q \ge 2$ are relatively prime integers, then $\Vir_{p, q}$ is classically free if and only if $q = 2$ by \cite{van_ekeren_chiral_2021}.
\end{example}

\begin{example}
  \label{exa:19}
  It was proven in \cite{andrews_singular_2022} that the Ising model $\Vir_{3, 4}$ is not classically free.
  In fact, by \cite[Theorem 2]{andrews_singular_2022},
  \begin{equation*}
    \ker(\phi_{\Vir_{3, 4}}) = (b)_{\partial},
  \end{equation*}
  where
  \begin{equation*}
    b = L_{-4}L_{-3}L_{-2} + \tfrac{1}{6}L_{-5}L_{-2}^2,
  \end{equation*}
  and $(b)_{\partial}$ is the differential ideal generated by $b$, cf.\ \Cref{sec:graded-jet-algebras} and \Cref{exa:14} where $\partial$ is denoted by $T$.
\end{example}

In algebra, commutative algebras are often required to be finitely generated.
In the theory of $\mathbb{N}$-graded vertex algebras, the equivalent of this is assuming that $V$ is finitely strongly generated.
Fortunately, these two concepts are related, as the following theorem shows.

\begin{theorem}[{\cite[Theorem 4.8]{li_abelianizing_2005}}]
  \label{thr:12}
  Let $S \subseteq V$ be a subset of homogeneous elements of a lower truncated $\mathbb{Z}$-graded vertex algebra $V$.
  The following are equivalent:
  \begin{enumerate}
  \item $S$ strongly generates $V$;
  \item $\sigma_0(S)$ generates $R_V$ as an algebra without the Poisson structure.
  \end{enumerate}
  In particular, $V$ is finitely strongly generated if and only if $R_V$ is finitely generated.
\end{theorem}

Let $V$ be a vertex algebra, and let $M$ be a module over $V$.
By \Cref{lmm:5}, we have
\begin{equation*}
  C_2M = \vspan\{a^M_{(-2)}u \mid a \in V, u \in M\} = F_1M.
\end{equation*}
We define the \emph{Zhu $C_2$-module of $M$} by
\begin{equation*}
  R_M = M/C_2M = F_0M/F_1M \subseteq \gr_F(M).
\end{equation*}
The fact that $\gr_F(M)$ is a module over $\gr_F(V)$ implies that $R_M$ is a module over $R_V$ with operations given as follows.
For $a \in V$ and $u \in M$, we set:
\begin{align*}
  \sigma_0(a)\sigma^M_0(u) &= \sigma^M_0(a^M_{(-1)}u), \\
  \{\sigma_0(a), \sigma^M_0(u)\} &= \sigma^M_0(a^M_{(0)}u).
\end{align*}

If $V$ is graded and $M$ is a graded $V$-module, then, as we explained in \Cref{sec:filtr-vert-algebr}, $\gr_F(M)$ is graded $\gr_F(V)$-module.
Thus, $R_M$ is also a graded $R_V$-module.

We have constructed a functor
\begin{equation*}
  R: \{\text{(graded) $V$-modules}\} \to \{\text{(graded) $R_V$-modules}\}.
\end{equation*}

A $V$-module $M$ is called \emph{$C_2$-cofinite} if $R_M$ is finite dimensional.
The condition of $C_2$-cofiniteness on modules has several implications, as we will see later.

\begin{example}[$R_{M(c, h)}$ and $R_{L(c, h)}$]
  \label{exa:20}
  We pick a highest weight $(c, h)$.
  As in \Cref{exa:16}, we consider $\mathbb{C}[L_{-2}]$ as the polynomial algebra in one variable $L_{-2}$, and we equip it with the trivial Poisson bracket.
  We consider $\bigoplus_{k \in \mathbb{N}}\mathbb{C}[L_{-2}]L_{-1}^k \cong \mathbb{C}[L_{-2}, L_{-1}]$ as a module over the Poisson algebra $\mathbb{C}[L_{-2}]$ with Poisson bracket given by $\{L_{-2}, L_{-1}^k\} = L_{-1}^{k + 1}$ for $k \in \mathbb{N}$.
  By \Cref{exa:15} and \Cref{exa:16}, we have the following isomorphism of modules over Poisson algebras
  \begin{align*}
    R_{M(c, h)} &\xrightarrow{\sim} \bigoplus_{k \in \mathbb{N}}\mathbb{C}[L_{-2}]L_{-1}^k, \\
    \sigma_0(L_{-2}|c, h\rangle) &\mapsto L_{-2}, \\
    \sigma_0(L_{-1}|c, h\rangle) &\mapsto L_{-1}.
  \end{align*}
  We have a natural epimorphism
  \begin{align*}
    \pi_{c, h}: M(c, h) &\twoheadrightarrow L(c, h), \\
    \pi_{c, h}(u) &= u + J(c, h),
  \end{align*}
  and it satisfies $\ker(\pi_{c, h}) = J(c, h)$.
  Applying the functor $R$, we obtain an epimorphism
  \begin{equation*}
    R_{\pi_{c, h}}: R_{M(c, h)} \twoheadrightarrow R_{L(c, h)}.
  \end{equation*}
  From the equation $\ker(R_{\pi_{(c, h)}}) = \sigma_0(J(c, h))$, we obtain
  \begin{equation*}
    R_{L(c, h)} \cong \frac{\bigoplus_{k \in \mathbb{N}}\mathbb{C}[L_{-2}]L_{-1}^k}{\sigma_0(J(c, h))}.
  \end{equation*}
\end{example}

\begin{theorem}[{\cite[Lemma 4.2]{li_abelianizing_2005}}]
  \label{thr:13}
  Let $V$ be a vertex algebra, and let $M$ be a $V$-module.
  As a $\gr_F(V)$-module without the Poisson structure, $\gr_F(M)$ is generated by $R_M$, i.e.,
  \begin{equation*}
    \gr_F(M) = \gr_F(V)R_M.
  \end{equation*}
\end{theorem}

Let $V$ be a (graded) vertex algebra, and let $M$ be a (graded) $V$-module.
We have a natural (graded) inclusion $\inc: R_M \hookrightarrow \gr_F(M)$ of (graded) $R_V$-modules.
As we explained in \Cref{sec:graded-vert-poiss}, we can consider $JR_V \otimes_{R_V} R_M$ as a (graded) $JR_V$-module.
Because we have a (graded) epimorphism of vertex Poisson algebras $\phi_V: JR_V \twoheadrightarrow \gr_F(V)$, we can consider the (graded) $\gr_F(V)$-module $\gr_F(M)$ as a (graded) $JR_V$-module.
By the universal property of the (graded) $JR_V$-module $JR_V \otimes_{R_V} R_M$ (see \Cref{sec:graded-vert-poiss}), there is a (graded) $JR_V$-module homomorphism $\phi_M: JR_V \otimes_{R_V} R_M \to \gr_F(M)$ such that the following diagram commutes
\begin{equation*}
  \begin{tikzcd}
    R_M \arrow[r, "\inc", hook] \arrow[rd, "\inc"', hook] & JR_V \otimes_{R_V} R_M \arrow[d, "\phi_M"] \\
    & \gr_F(M)
  \end{tikzcd}
\end{equation*}

\begin{lemma}
  \label{lmm:8}
  The (graded) $JR_V$-module homomorphism $\phi_M: JR_V \otimes_{R_V} R_M \to \gr_F(M)$ defined above is surjective.
\end{lemma}

\begin{proof}
  The assertion follows from \Cref{lmm:7} and \Cref{thr:13}.
\end{proof}

When $\phi_M$ is an isomorphism, we say the $M$ is \emph{classically free}.

\begin{example}[$JR_{\Vir^c} \otimes_{R_{\Vir^c}} R_{M(c, h)}$]
  \label{exa:21}
  We pick a highest weight $(c, h)$.
  The Verma module $M(c, h)$ is always classically free as a $\Vir^c$-module because by \Cref{exa:15}, \Cref{exa:16} and \Cref{exa:20},
  \begin{align*}
    JR_{\Vir^c} \otimes_{R_{\Vir^c}} R_{M(c, h)} &\cong \mathbb{C}[L_{-2}, L_{-3}, \dots] \otimes_{\mathbb{C}[L_{-2}]} \bigoplus_{k \in \mathbb{N}}\mathbb{C}[L_{-2}]L_{-1}^k \\
                                                 &\cong \bigoplus_{k \in \mathbb{N}}\mathbb{C}[L_{-2}, L_{-3}, \dots]L_{-1}^k \\
                                                 &\cong \gr_F(M(c, h)).
  \end{align*}
\end{example}

We wish to determine when the irreducible modules $L(c_{p, q}, h_{m, n})$ over the simple Virasoro vertex algebras $\Vir_{p, q}$ are classically free, as mentioned in \Cref{thr:9}.

\begin{theorem}[{\cite[Proposition 4.12]{li_abelianizing_2005}}]
  \label{thr:14}
  Let $V$ be a vertex operator algebra, let $M$ be an admissible $V$-module, and let $T \subseteq M$ be a subset of homogeneous elements of $M$.
  The following are equivalent:
  \begin{enumerate}
  \item $M$ is strongly generated over $V$ by $T$;
  \item $\sigma^M_0(T)$ generates $R_M$ as a module over $R_V$ without the Poisson structure.
  \end{enumerate}
  In particular, $M$ is finitely strongly generated over $V$ if and only if $R_M$ is a finitely generated $R_V$-module.
\end{theorem}

\section{Proofs of the main theorems}
\label{sec:proofs-main-theorems}

In this section, we fix $s \in \mathbb{Z}_+$ and $i \in \{1, \dots, s\}$.

A partition $\lambda = [\lambda_1, \dots, \lambda_m]$ \emph{contains a partition} $\eta = [\eta_1, \dots, \eta_n]$, written as $\eta \subseteq \lambda$, if $m \ge n$ and there is $i \in \mathbb{Z}_+$ such that $1 \le i \le m - n + 1$ and $[\lambda_i, \lambda_{i + 1}, \dots, \lambda_{i + n - 1}] = \eta$.

We define
\begin{equation*}
  p^{s, i}(t, q) = \sum_{\lambda \in P^{s, i}}t^{\len(\lambda)}q^{\Delta(\lambda)} \in \mathbb{C}[[t, q]],
\end{equation*}
where $P^{s, i}$ is the set of partitions that do not contain any partition in $R^{s, i}$ as defined in \Cref{sec:introduction}, i.e.,
\begin{equation*}
  P^{s, i} = \{\lambda \vdash \mid \text{for $\eta \in R^{s, i}$, $\lambda \nsupseteq \eta$}\}.
\end{equation*}

\begin{lemma}
  \label{lmm:9}
  We have
  \begin{equation*}
    p^{s, i}(t, q) = J_{s, i}(0, t, q) = \sum_{k = (k_1, \dots, k_{s - 1}) \in \mathbb{N}^{s - 1}}t^{kB^{(s)}_{s - 1}}\frac{q^{\frac{1}{2}kG^{(s)}k^T + kB^{(s)}_{s - i}}}{(q)_{k_1}\dots(q)_{k_{s - 1}}}.
  \end{equation*}
\end{lemma}

\begin{proof}
  Following the notation of \cite[\S7.2]{andrews_theory_1998}, we write:
  \begin{align*}
    p^{s, i}(t, q) &= \sum_{m = 0}^{\infty}\sum_{n = 0}^{\infty}b_{s, i}(m, n)t^mq^n, \\
    J_{s, i}(0, t, q) &= \sum_{m = 0}^{\infty}\sum_{n = 0}^{\infty}c_{s, i}(m, n)t^mq^n.
  \end{align*}
  By \cite[\S7.3]{andrews_theory_1998}, we have $b_{s, i}(m, n) = c_{s, i}(m, n)$ for $m, n \in \mathbb{N}$.
  This means $p^{s, i}(t, q) = J_{s, i}(0, t, q)$.

  By the proof of \cite[Theorem 7.8]{andrews_theory_1998}, we have
  \begin{equation*}
    J_{s, i}(0, t, q) = \sum_{k = (k_1, \dots, k_{s - 1}) \in \mathbb{N}^{s - 1}}\frac{t^{N_1 + \dots + N_{s - 1}}q^{N_1^2 + \dots + N_{s - 1}^2 + N_i + \dots + N_{s - 1}}}{(q)_{k_1}\dots(q)_{k_{s - 1}}},
  \end{equation*}
  where $N_j = k_j + \dots + k_{s - 1}$ for $j = 0, 1, \dots, s - 1$.
  Since
  \begin{align*}
    N_1^2 + \dots + N_{s - 1}^2 &= \frac{1}{2}kG^{(s)}k^T, \\
    N_j + \dots + N_{s - 1} &= kB^{(s)}_{s - j} \quad \text{for $j = 1, \dots, s - 1$},
  \end{align*}
  we have
  \begin{equation*}
    \sum_{k = (k_1, \dots, k_{s - 1}) \in \mathbb{N}^{s - 1}}\frac{t^{N_1 + \dots + N_{s - 1}}q^{N_1^2 + \dots + N_{s - 1}^2 + N_i + \dots + N_{s - 1}}}{(q)_{k_1}\dots(q)_{k_{s - 1}}} = \sum_{k = (k_1, \dots, k_{s - 1}) \in \mathbb{N}^{s - 1}}t^{kB^{(s)}_{s - 1}}\frac{q^{\frac{1}{2}kG^{(s)}k^T + kB^{(s)}_{s - i}}}{(q)_{k_1}\dots(q)_{k_{s - 1}}}. \qedhere
  \end{equation*}
\end{proof}

\begin{lemma}
  \label{lmm:10}
  We have
  \begin{equation*}
    \ch_{L(c_{2, 2s + 1}, h_{1, i})}(q) = q^{h_{1, i}}p^{s, i}(1, q) = q^{h_{1, i}}\left(\prod^{\infty}_{\substack{n = 1 \\ n \not\equiv 0, \pm i \mod 2k + 1}}(1 - q^n)^{-1}\right).
  \end{equation*}
\end{lemma}

\begin{proof}
  By \Cref{lmm:9}, \cite[Lemma 7.3]{andrews_theory_1998} and \cite[Corollary 2.9]{andrews_theory_1998}, we have
  \begin{align*}
    q^{h_{1, i}}p^{s, i}(1, q) &= q^{h_{1, i}}J_{s, i}(0, 1, q) \\
                               &= q^{h_{1, i}}\left(\prod^{\infty}_{\substack{n = 1 \\ n \not\equiv 0, \pm i \mod 2k + 1}}(1 - q^n)^{-1}\right) \\
                               &= \frac{q^{h_{1, i}}}{(q)_{\infty}}\sum_{k \in \mathbb{N}}(-1)^kq^{(2s + 1)k(k + 1)/2 - ki}(1 - q^{(2k + 1)i}) \\
                               &= \frac{q^{h_{1, i}}}{(q)_{\infty}}\sum_{k \in \mathbb{Z}}(-1)^kq^{(2s + 1)k(k + 1)/2 - ki}.
  \end{align*}
  The disjoint union $\mathbb{Z} = \{2k \mid k \in \mathbb{Z}\} \cup \{-2k - 1 \mid k \in \mathbb{Z}\}$ and \Cref{thr:10} imply that
  \begin{align*}
    \frac{q^{h_{1, i}}}{(q)_{\infty}}\sum_{k \in \mathbb{Z}}(-1)^kq^{(2s + 1)k(k + 1)/2 - ki} &= \frac{q^{h_{1, i}}}{(q)_{\infty}}\Bigg(\sum_{k \in \mathbb{Z}}(-1)^{2k}q^{(2s + 1)2k(2k + 1)/2 - 2ki} \\
                                                                                              &\quad + \sum_{k \in \mathbb{Z}}(-1)^{-2k - 1}q^{(2s + 1)(-2k - 1)(-2k - 1 + 1)/2 - (-2k - 1)i}\Bigg) \\
                                                                                              &= \frac{q^{h_{1, i}}}{(q)_{\infty}}\left(\sum_{k \in \mathbb{Z}}q^{2k^2(2s + 1) + k(2s + 1 - 2i)} - \sum_{k \in \mathbb{Z}}q^{2k^2(2s + 1) + k(2s + 2i + 1) + i}\right) \\
                                                                                              &= \ch_{L(c_{2, 2s + 1}, h_{1, i})}(q). \qedhere
  \end{align*}
\end{proof}

We order the PBW basis of $U(\Vir^-) = \vspan\{L_{\lambda} \mid \lambda \vdash\}$ by degree reverse lexicographic order with $L_{-1} > L_{-2} > \dots$.
Formally, for any partitions $\lambda$ and $\eta$, we define
\begin{equation*}
  L_{\lambda} \le L_{\eta}\text{ if and only if }p_{\lambda} \le p_{\eta}.
\end{equation*}
For $x \in U(\Vir^-)$ with $x \neq 0$, we may write
\begin{equation*}
  x = c_1L_{\lambda^1} + c_2L_{\lambda^2} + \dots + c_rL_{\lambda^r},
\end{equation*}
where for $1 \le i \le r$, $0 \neq c_i \in \mathbb{C}$ and $L_{\lambda^1} > L_{\lambda^2} > \dots > L_{\lambda^r}$.
We define the \emph{leading power of $x$} as $\lp(x) = L_{\lambda^1}$.
We set $\lp(0) = 0$.
Next, for a highest weight $(c, h)$, we extend the definition of $\lp$ from $U(\Vir^-)$ to $M(c, h)$ by considering the isomorphism of vector spaces $U(\Vir^-) \xrightarrow{\sim} M(c, h), L_{\lambda} \mapsto L_{\lambda}|c, h\rangle$, where $\lambda$ is a partition.

\begin{remark}
  \label{rmk:4}
  The definition of the order in the PBW basis of $U(\Vir^-)$ was made so that for a highest weight $(c, h)$, a partition $\lambda$ and $u \in M(c, h)$, if $\lp(u) = L_{\lambda}|c, h\rangle$, then $\lp(\gamma^{\len(\lambda)}(u)) = p_{\lambda}$.
\end{remark}

\begin{lemma}
  \label{lmm:11}
  For $\lambda \in R^{s, i}$, there exists $u \in J(c_{2, 2s + 1}, h_{1, i})$ homogeneous such that $\lp(u) = L_{\lambda}|c_{2, 2s + 1}, h_{1, i}\rangle$.
\end{lemma}

\begin{proof}
  By \Cref{thr:7}, the maximal proper ideal of $\Vir^{c_{2, 2s + 1}}$ is generated by a singular vector of conformal weight $2s$, denoted by $a_{2, 2s + 1}$, which has the form
  \begin{equation}
    \label{eq:7}
    a_{2, 2s + 1} = L_{-2}^s\vac + \sum_{\substack{\mu \vdash 2s \\ \mu_{\len(\mu)} \ge 2 \\ \len(\mu) < s}}c_{\mu}L_{\mu}\vac,
  \end{equation}
  where $c_{\mu} \in \mathbb{Q}$.
  We have two cases:
  \begin{enumerate}
  \item $\lambda = [\underbrace{a + 1, \dots, a + 1}_{s - d}, \underbrace{a, \dots, a}_d]$ for some $a \in \mathbb{Z}_+$ and $d \in \{1, \dots, s\}$.
    We set
    \begin{equation*}
      u = (a_{2, 2s + 1})_{(s - sa + d - 1)}|c_{2, 2s + 1}, h_{1, i}\rangle
    \end{equation*}
    and claim
    \begin{equation*}
      \lp(u) = L_{\lambda}|c_{2, 2s + 1}, h_{1, i}\rangle.
    \end{equation*}
    To prove this, we first prove that
    \begin{equation}
      \label{eq:8}
      (L_{-2}^s\vac)_{(s - sa + d - 1)}|c_{2, 2s + 1}, h_{1, i}\rangle = \binom{s}{d}L_{\lambda}|c_{2, 2s + 1}, h_{1, i}\rangle + \sum_{\substack{\mu \vdash \Delta(\lambda) \\ \len(\mu) \le \len(\lambda) \\ \mu \neq \lambda}}c_{\mu}L_{\mu}|c_{2, 2s + 1}, h_{1, i}\rangle,
    \end{equation}
    where $c_{\mu} \in \mathbb{Q}$.
    Applying \Cref{prp:3} and \Cref{lmm:2} with
    \begin{align*}
      a^1(z) = \dots = a^s(z) &= L(z) = \sum_{n \in \mathbb{Z}}L_{(n)}z^{-n - 1} = \sum_{n \in \mathbb{Z}}L_nz^{-n - 2}, \\
      b &= |c_{2, 2s + 1}, h_{1, i}\rangle, \\
      l &= s - sa + d - 1,
    \end{align*}
    we obtain
    \begin{equation}
      \label{eq:9}
      (L_{-2}^s\vac)_{(l)}|c_{2, 2s + 1}, h_{1, i}\rangle = \sum_{n_1, \dots, n_{s - 1} \in \mathbb{N}}\sum_{k = 0}^{s - 1}R^{l, k}_{n_1, \dots, n_{s - 1}}(L(z), \dots, L(z))|c_{2, 2s + 1}, h_{1, i}\rangle,
    \end{equation}
    where $R^{l, k}_{n_1, \dots, n_{s - 1}}(L(z), \dots, L(z))$ is the sum of $\binom{s - 1}{k}$ terms given by
    \begin{align*}
      &R^{l, k}_{n_1, \dots, n_{s - 1}}(L(z), \dots, L(z)) = \\
      &\sum_{\substack{1 \le i_1 < \dots < i_k \le s - 1 \\ 1 \le j_1 < \dots < j_{s - 1 - k} \le s - 1 \\ \{i_1, \dots, i_k\} \cup \{j_1, \dots, j_{s - 1 - k}\} = \{1, \dots, s - 1\}}}L_{(-n_{j_1} - 1)}\dots L_{(-n_{j_{s - 1 - k}} - 1)}L_{(l - k - \sum_{r = 1}^kn_{i_r} + \sum_{r = 1}^{s - 1 - k}n_{j_r})}L_{(n_{i_k})}\dots L_{(n_{i_1})},
    \end{align*}
    where
    \begin{align*}
      &L_{(-n_{j_1} - 1)}\dots L_{(-n_{j_{s - 1 - k}} - 1)}L_{(l - k - \sum_{r = 1}^kn_{i_r} + \sum_{r = 1}^{s - 1 - k}n_{j_r})}L_{(n_{i_k})}\dots L_{(n_{i_1})} = \\
      &L_{-n_{j_1} - 2}\dots L_{-n_{j_{s - 1 - k}} - 2}L_{s - sa + d - 2 - k - \sum_{r = 1}^kn_{i_r} + \sum_{r = 1}^{s - 1 - k}n_{j_r}}L_{n_{i_k} - 1}\dots L_{n_{i_1} - 1}
    \end{align*}
    We see that each term $L_{\mu}$ appearing in $R^{l, k}_{n_1, \dots, n_{s - 1}}(L(z), \dots, L(z))$, where $\mu$ is a combination (not necessarily a partition), satisfies $\Delta(\mu) = sa + s - d = \Delta(\lambda)$ and $\len(\mu) \le \len(\lambda)$.

    We recall that $L_{\lambda} - L_{\lambda\sigma} \in U(\Vir^{-})^{\len(\lambda) - 1}$ for a combination $\lambda$ and a permutation $\sigma \in S_m$ (see \cite[\S2]{dixmier_enveloping_1996} for details).
    Therefore, for a combination $\mu$, we can expand $L_{\mu}|c_{2, 2s + 1}, h_{1, i}\rangle$ in the following way
    \begin{equation*}
      L_{\mu}|c_{2, 2s + 1}, h_{1, i}\rangle = L_{\prt(\mu)}|c_{2, 2s + 1}, h_{1, i}\rangle + \sum_{\substack{\mu \vdash \Delta(\lambda) \\ \len(\mu) < \len(\lambda)}}c_{\mu}L_{\mu}|c_{2, 2s + 1}, h_{1, i}\rangle,
    \end{equation*}
    where $c_{\mu} \in \mathbb{Q}$.
    We need to compute the coefficient of $L_{\lambda}|c_{2, 2s + 1}, h_{1, i}\rangle$ in $u$ when expressed as a linear combination of the elements of the usual PBW basis.
    To do this, we need to see how much each sum $R^{l, k}_{n_1, \dots, n_{s - 1}}(L(z), \dots, L(z))|c_{2, 2s + 1}, h_{1, i}\rangle$ contributes to the coefficient of $L_{\lambda}|c_{2, 2s + 1}, h_{1, i}\rangle$ in $u$.
    We have two subcases:
    \begin{description}[leftmargin = !]
    \item[Subcase $a = 1$] In the sum \eqref{eq:9}, if $k > d$ or $k < d - 1$ or $n_i \neq 0$ for some $i = 1, \dots, s - 1$, then $R^{l, k}_{n_1, \dots, n_{s - 1}}(L(z), \dots, L(z))|c_{2, 2s + 1}, h_{1, i}\rangle$ contributes $0$.
      As for the two remaining sums, the sum $R^{l, d - 1}_{0, \dots, 0}(L(z), \dots, L(z))|c_{2, 2s + 1}, h_{1, i}\rangle$ contributes $\binom{s - 1}{d - 1}$ because we are picking $d - 1$ elements from $\{1, \dots, s - 1\}$, and the sum $R^{l, d}_{0, \dots, 0}(L(z), \dots, L(z))|c_{2, 2s + 1}, h_{1, i}\rangle$ contributes $\binom{s - 1}{d}$ because we are picking $d$ elements from $\{1, \dots, s - 1\}$.
      Since $\binom{s - 1}{d - 1} + \binom{s - 1}{d} = \binom{s}{d}$, we obtain \eqref{eq:8}.
    \item[Subcase $a > 1$] In the sum \eqref{eq:9}, if $k > 0$, then $R^{l, k}_{n_1, \dots, n_{s - 1}}(L(z), \dots, L(z))|c_{2, 2s + 1}, h_{1, i}\rangle$ contributes $0$.
      We have
      \begin{equation*}
        R^{l, 0}_{n_1, \dots, n_{s - 1}}(L(z), \dots, L(z)) = L_{-n_1 - 2}\dots L_{-n_{s - 1} - 2}L_{s - sa + d - 2 + \sum_{r = 1}^{s - 1}n_r}.
      \end{equation*}
      Therefore, the sum $R^{l, 0}_{n_1, \dots, n_{s - 1}}(L(z), \dots, L(z))|c_{2, 2s + 1}, h_{1, i}\rangle$ contributes $\binom{s - 1}{d - 1}$ when exactly $d - 1$ of the $n_i$s are equal to $a - 2$ and $s - d$ of the $n_i$s are equal to $a - 1$, contributes $\binom{s - 1}{d}$ when exactly $d$ of the $n_i$s are equal to $a - 2$ and $s - d - 1$ of the $n_i$s are equal to $a - 1$, and contributes $0$ otherwise.
      Since $\binom{s - 1}{d - 1} + \binom{s - 1}{d} = \binom{s}{d}$, we obtain \eqref{eq:8}.
    \end{description}
  \item $\lambda = \underbrace{[1, 1, \dots, 1]}_i$.
    This follows from \cite{benoit_degenerate_1988}.
  \end{enumerate}
  We note that if $\mu$ is a partition of $\Delta(\lambda)$ such that $\len(\mu) \le \len(\lambda)$ and $\mu \neq \lambda$, then $L_{\lambda} > L_{\mu}$.
  Thus, \eqref{eq:8} implies $\lp((L_{-2}^s\vac)_{(s - sa + d - 1)}|c_{2, 2s + 1}, h_{1, i}\rangle) = L_{\lambda}|c_{2, 2s + 1}, h_{1, i}\rangle$.
  From \eqref{eq:7} and \Cref{lmm:2}, we see that $\lp(u) = \lp((L_{-2}^s\vac)_{(s - sa + d - 1)}|c_{2, 2s + 1}, h_{1, i}\rangle) = L_{\lambda}|c_{2, 2s + 1}, h_{1, i}\rangle$.
  Finally, by \Cref{thr:9}, $u \in J(c_{2, 2s + 1}, h_{1, i})$, and it is clear that $u$ is homogeneous from its definition.
\end{proof}

\begin{remark}
  \label{rmk:5}
  For any partitions $\lambda$ and $\eta$, if $\lambda \supseteq \eta$, then $p_{\eta} \mid p_{\lambda}$.
  The converse is not true.
  For example, $p_{[4, 2]} \mid p_{[4, 3, 2]}$, but $[4, 3, 2] \nsupseteq [4, 2]$.
  However, if $\eta = [\eta_1, \dots, \eta_m]$ and $\eta_1 - \eta_m \le 1$, then $\lambda \supseteq \eta$ if and only if $p_{\eta} \mid p_{\lambda}$.
\end{remark}

\begin{lemma}
  \label{lmm:12}
  There is an alternative description for $P$, namely
  \begin{equation*}
    P^{s, i} = \{\lambda \vdash \mid \text{for $\eta \in R^{s, i}$, $p_{\eta} \nmid p_{\lambda}$}\}.
  \end{equation*}
\end{lemma}

\begin{proof}
  This is a consequence of \Cref{rmk:5}.
\end{proof}

\begin{proof}[Proof of \Cref{thr:2}]
  This follows from Gröbner basis theory, \Cref{prp:1}, \Cref{lmm:10}, \Cref{rmk:4}, \Cref{lmm:11} and \Cref{lmm:12}, see \cite{salazar_pbw_2024} for details.
\end{proof}

\begin{proof}[Proof of \Cref{thr:1}]
  This is a corollary of \Cref{thr:2}, see \cite{salazar_pbw_2024} for details.
\end{proof}

\begin{proof}[Proof of \Cref{thr:3}]
  By \Cref{exa:20}, we have
  \begin{equation*}
    R_{L(c_{2, 2s + 1}, h_{1, i})} \cong \frac{\bigoplus_{k \in \mathbb{N}}\mathbb{C}[L_{-2}]L_{-1}^k}{\sigma_0(J(c_{2, 2s + 1}, h_{1, i}))} \cong \frac{\mathbb{C}[L_{-2}, L_{-1}]}{\sigma_0(J(c_{2, 2s + 1}, h_{1, i}))}.
  \end{equation*}
  First, we prove that for a partition $\eta$ with $\eta = \emptyset$ or $2 \ge \eta_1$,
  \begin{equation*}
    p_{\eta} + \sigma_0(J(c_{2, 2s + 1}, h_{1, i})) \in \vspan\{p_{\lambda} + \sigma_0(J(c_{2, 2s + 1}, h_{1, i})) \mid \text{$\lambda = \emptyset$ or ($\lambda \in P^{s, i}$ and $2 \ge \lambda_1$)}\} = W.
  \end{equation*}
  We assume, for the sake of contradiction, that there is a power $p_{\eta}$ with $\eta = \emptyset$ or $2 \ge \eta_1$ not satisfying $p_{\eta} + \sigma_0(J(c_{2, 2s + 1}, h_{1, i})) \in W$.

  Let $p_{\eta_0}$ be the smallest of such powers in the degree reverse lexicographic order with $L_{-1} > L_{-2} > \dots$.
  We must have $\eta_0 \neq \emptyset$ and $\eta_0 \notin P^{s, i}$, otherwise $p_{\eta_0} + \sigma_0(J(c_{2, 2s + 1}, h_{1, i})) \in W$.
  By \Cref{lmm:12}, there is $\mu \in R^{s, i}$ such that $p_{\mu} \mid p_{\eta_0}$.
  From \eqref{eq:2}, \Cref{rmk:4}, \Cref{lmm:11} and the fact that $\lp$ is multiplicative, we obtain $u \in J(c_{2, 2s + 1}, h_{1, i})$ such that $\lp(u) = L_{\eta_0}|c_{2, 2s + 1}, h_{1, i}\rangle$.

  From the equation
  \begin{equation*}
    u = L_{\eta_0}|c_{2, 2s + 1}, h_{1, i}\rangle + \sum_{\substack{\mu \vdash \Delta(\eta_0) \\ p_{\mu} < p_{\eta_0}}}c_{\mu}L_{\mu}|c_{2, 2s + 1}, h_{1, i}\rangle,
  \end{equation*}
  where $c_{\mu} \in \mathbb{Q}$, we obtain
  \begin{equation*}
    p_{\eta_0} + \sigma_0(J(c_{2, 2s + 1}, h_{1, i}))= -\sum_{\substack{\mu \vdash \Delta(\eta_0) \\ p_{\mu} < p_{\eta_0} \\ 2 \ge \mu_1} }c_{\mu}(p_{\mu} + \sigma_0(J(c_{2, 2s + 1}, h_{1, i}))).
  \end{equation*}
  The last equation implies $p_{\eta_0} + \sigma_0(J(c_{2, 2s + 1}, h_{1, i})) \in W$ because $p_{\mu} + \sigma_0(J(c_{2, 2s + 1}, h_{1, i})) \in W$ if $p_{\mu} < p_{\eta_0}$ by the minimality of $\eta_0$, and this is a contradiction.
  We have proved that
  \begin{equation}
    \label{eq:10}
    \frac{\mathbb{C}[L_{-2}, L_{-1}]}{\sigma_0(J(c_{2, 2s + 1}, h_{1, i}))} = \vspan\{p_{\lambda} + \sigma_0(J(c_{2, 2s + 1}, h_{1, i})) \mid \text{$\lambda = \emptyset$ or ($\lambda \in P^{s, i}$ and $2 \ge \lambda_1$)}\}
  \end{equation}

  By \cite{van_ekeren_chiral_2021} or \cite{bruschek_arc_2013}, we already know $\Vir_{2, 2s + 1}$ is classically free.
  This means that $\gr_F(\Vir_{2, 2s + 1}) \cong JR_{\Vir_{2, 2s + 1}} \cong J(\mathbb{C}[L_{-2}]/(L_{-2}^s)) \cong \mathbb{C}[L_{-2}, L_{-3}, \dots]/(L_{-2}^s)_{\partial}$ and
  \begin{equation}
    \label{eq:11}
    \mathbb{C}[L_{-2}, L_{-3}, \dots]/(L_{-2}^s)_{\partial} = \bigoplus_{\lambda \in P^{s, 1}}\mathbb{C}(p_{\lambda} + (L_{-2}^s)_{\partial}).
  \end{equation}

  We have
  \begin{equation*}
    JR_{\Vir_{2, 2s + 1}} \otimes_{R_{\Vir_{2, 2s + 1}}} R_{L(c_{2, 2s + 1}, h_{1, i})} \cong \frac{\mathbb{C}[L_{-2}, L_{-3}, \dots]}{(L_{-2}^s)_{\partial}} \otimes_{\frac{\mathbb{C}[L_{-2}]}{(L_{-2}^s)}}\frac{\mathbb{C}[L_{-2}, L_{-1}]}{\sigma_0(J(c_{2, 2s + 1}, h_{1, i}))}
  \end{equation*}

  Given a partition $\lambda \in P^{s, i}$, we can write $\lambda$ uniquely as $\lambda = [\lambda^1, \lambda^2]$, where $\lambda^1 \in P^{s, 1}$ satisfies $\lambda^1 = \emptyset$ or $\lambda^1_{\len(\lambda^1)} \ge 3$ and $\lambda^2 = \emptyset$ or ($\lambda^2 \in P^{s, i}$ and $2 \ge \lambda^2_1$).

  Given $\lambda \in P^{s, i}$, we define
  \begin{equation*}
    t_{\lambda} = (p_{\lambda^1} + (L_{-2}^s)_{\partial})\otimes(p_{\lambda^2} + \sigma_0(J(c_{2, 2s + 1}, h_{1, i}))) \in \frac{\mathbb{C}[L_{-2}, L_{-3}, \dots]}{(L_{-2}^s)_{\partial}} \otimes_{\frac{\mathbb{C}[L_{-2}]}{(L_{-2}^s)}} \frac{\mathbb{C}[L_{-2}, L_{-1}]}{\sigma_0(J(c_{2, 2s + 1}, h_{1, i}))}.
  \end{equation*}
  From \eqref{eq:10} and \eqref{eq:11}, we conclude that
  \begin{equation}
    \label{eq:12}
    \frac{\mathbb{C}[L_{-2}, L_{-3}, \dots]}{(L_{-2}^s)_{\partial}} \otimes_{\frac{\mathbb{C}[L_{-2}]}{(L_{-2}^s)}} \frac{\mathbb{C}[L_{-2}, L_{-1}]}{\sigma_0(J(c_{2, 2s + 1}, h_{1, i}))} = \vspan\{t_{\lambda} \mid \lambda \in P^{s, i}\}.
  \end{equation}
  By \Cref{prp:1} with $\gr_F$ instead of $\gr$, \Cref{lmm:8} and \Cref{thr:2}, the natural epimorphism $\phi_{L(c_{2, 2s + 1}, h_{1, i})}: JR_{\Vir_{2, 2s + 1}} \otimes_{R_{\Vir_{2, 2s + 1}}} R_{L(c_{2, 2s + 1}, h_{1, i})} \twoheadrightarrow \gr_F(L(c_{2, 2s + 1}, h_{1, i}))$ respects the grading while mapping the spanning set $\{t_{\lambda} \mid \lambda \in P^{s, i}\}$ onto a basis of $\gr_F(L(c_{2, 2s + 1}, h_{1, i}))$.
  Moreover, $\dim((JR_{\Vir_{2, 2s + 1}} \otimes_{R_{\Vir_{2, 2s + 1}}} R_{L(c_{2, 2s + 1}, h_{1, i})})_{h_{1, i} + n})< \infty$ and $\dim(\gr_F(L(c_{2, 2s + 1}, h_{1, i}))_{h_{1, i} + n})< \infty$ for $n \in \mathbb{N}$.
  These observations, together with \eqref{eq:12}, lead us to conclude that the natural epimorphism $\phi_{L(c_{2, 2s + 1}, h_{1, i})}$ is actually an isomorphism.
\end{proof}

\appendix
\section{Graded Poisson algebras and their modules}
\label{sec:grad-poiss-algeb}

Let $A$ be a commutative associative algebra.
A \emph{Hamiltonian operator of $A$} is a diagonalizable operator $H \in \End(A)$ such that
\begin{equation*}
  H(ab) = H(a)b + aH(b) \quad \text{for $a, b \in A$}.
\end{equation*}
Thus, a Hamiltonian of $A$ is just a derivation of $A$.
An algebra with a Hamiltonian is called \emph{graded}.

Let $A$ be a differential commutative associative algebra with derivation $\partial$.
A \emph{Hamiltonian operator of $A$} is a Hamiltonian of $A$ as a commutative associative algebra such that
\begin{equation*}
  [H, \partial] = \partial.
\end{equation*}
It is possible to prove inductively that the last equation implies
\begin{equation}
  \label{eq:13}
  H\partial^n = n\partial^n + \partial^nH \quad \text{for $n \in \mathbb{N}$}.
\end{equation}

Let $A$ be a Poisson algebra as defined in \cite{caressa_examples_2003}.
A \emph{Hamiltonian operator of $A$} is a Hamiltonian $H$ of $A$ as a commutative associative algebra such that
\begin{equation*}
  H(\{a, b\}) = \{a, H(b)\} + \{H(a), b\} - \{a, b\} \quad \text{for $a, b \in A$}.
\end{equation*}

Let $A$ be a graded commutative associative algebra with Hamiltonian $H$, and let $M$ be an $A$-module.
If $A$ has a unit $1$, we further assume that $1u = u$ for $u \in M$.
A \emph{Hamiltonian operator of $M$} is a diagonalizable operator $H^M \in \End(M)$ such that
\begin{equation*}
  H^M(au) = H(a)u + aH^M(u) \quad \text{for $a \in A$ and $u \in M$}.
\end{equation*}

Let $A$ be a Poisson algebra.
A \emph{module over $A$} is an $A$-module $M$ in the usual associative sense equipped with a bilinear map $\{\bullet, \bullet\}: A \times M \to M$, which makes $M$ a Lie algebra module over $A$ such that for $a, b \in A$ and $u \in M$:
\begin{enumerate}
\item (Left Leibniz rule) $\{a, bu\} = \{a, b\}u + b\{a, u\}$;
\item (Right Leibniz rule) $\{ab, u\} = a\{b, u\} + b\{a, u\}$.
\end{enumerate}

Let $A$ be a graded Poisson algebra with Hamiltonian $H$, and let $M$ be an $A$-module.
A \emph{Hamiltonian operator of $M$} is a Hamiltonian $H^M$ of $M$ as a module over $A$ as a commutative associative algebra such that
\begin{equation*}
  H^M(\{a, u\}) = \{a, H^M(u)\} + \{H(a), u\} - \{a, u\} \quad \text{for $a \in A$ and $u \in M$}.
\end{equation*}

\section{Graded Jet algebras and their modules}
\label{sec:graded-jet-algebras}

In this appendix, by an algebra we will mean a commutative associative algebra with unit.
Let $R$ be a finitely generated algebra.
We now construct a differential algebra $JR$ called the \emph{jet algebra of $R$} and an algebra inclusion $\inc: R \hookrightarrow JR$ universal with this property, i.e., for a differential algebra $A$ and an algebra homomorphism $\phi: R \to A$, there is a unique differential algebra homomorphism $\overline{\phi}: JR \to A$ such that the following diagram commutes
\begin{equation*}
  \begin{tikzcd}
    R \arrow[r, "\inc", hook] \arrow[rd, "\phi"'] & JR \arrow[d, "\overline{\phi}"] \\
    & A
  \end{tikzcd}
\end{equation*}

Assuming $R = \mathbb{C}[x^1, \dots, x^r]/(f_1, \dots, f_s)$ for some polynomials $f_1, \dots, f_s \in \mathbb{C}[x^1, \dots, x^r]$, the construction is as follows.
We introduce new variables $x^j_{(-i)}$ for $i = 1, 2, \dots$, $j = 1, \dots, r$ and a derivation $\partial$ of the polynomial algebra $\mathbb{C}[x^j_{(-i)} \mid i = 1, 2, \dots, j = 1, \dots, r]$ by setting
\begin{equation*}
  \partial x^j_{(-i)} = ix^j_{(-i - 1)} \quad \text{for $i = 1, 2, \dots$, $j = 1, \dots, r$}.
\end{equation*}
We set (identifying $x^j$ with $x^j_{(-1)}$ when considering $f_i$ in the following equation)
\begin{align*}
  JR &= \mathbb{C}[x^j_{(-i)} \mid i = 1, 2, \dots, j = 1, \dots, r]/(\partial^jf_i \mid i = 1, \dots, s, j = 0, 1, \dots) \\
     &= \mathbb{C}[x^j_{(-i)} \mid i = 1, 2, \dots, j = 1, \dots, r]/(f_1, \dots, f_s)_{\partial},
\end{align*}
where the subscript $\partial$ indicates the differential subalgebra generated by the given subset.
By our definitions, $\partial$ factors through a derivation of $JR$, and we have an algebra inclusion $\inc: R \hookrightarrow JR, \inc(x^j + (f_1, \dots, f_s)) = x^j_{(-1)} + (f_1, \dots, f_s)_{\partial}$ for $j = 1, \dots, r$.
The fact that $\inc: R \hookrightarrow JR$ satisfies our desired universal property is explained in \cite[\S2.3]{arakawa_remark_2012} and \cite{ein_jet_2008}.

\begin{remark}
  \label{rmk:6}
  We see that the classes of the original variables $x^j$ generate $JR$ as a differential algebra, i.e.,
  \begin{equation*}
    JR = (x^1 + (f_1, \dots, f_s)_{\partial}, \dots, x^r + (f_1, \dots, f_s)_{\partial})_{\partial}.
  \end{equation*}
\end{remark}

If $f: R_1 \to R_2$ is a homomorphism of finitely generated algebras, then $Jf: JR_1 \to JR_2$ is defined by requiring that $Jf(\partial_1^n(\inc_1(x))) = \partial^n_2(\inc_2(f(x)))$ for $x \in R_1$ and $n \in \mathbb{N}$ (c.f.\ \Cref{rmk:6}).
For a finitely generated algebra $R$ and a differential algebra $A$, we have a natural isomorphism
\begin{equation*}
  \Hom_{\{\text{differential algebras}\}}(JR, A) \cong \Hom_{\{\text{algebras}\}}(R, A).
\end{equation*}

\begin{remark}
  \label{rmk:7}
  It is not true that we have defined a functor $J$ which is left adjoint to the forgetful functor $\{\text{differential algebras}\} \to \{\text{finitely generated algebras}\}$ because a differential algebra is generally not finitely generated.

  But we can easily work in the general case as follows.
  Let $R$ be an algebra (not necessarily finitely generated), and we consider the polynomial algebra in $R$ variables $\mathbb{C}[(x^j)_{j \in R}]$.
  We have a natural epimorphism $\pi: \mathbb{C}[(x^j)_{j \in R}] \twoheadrightarrow R, \pi(x^j) = j$ for $j \in R$.
  We can repeat the construction of $JR$ with $\mathbb{C}[(x^j)_{j \in R}]$ in place of $\mathbb{C}[x^1, \dots, x^r]$ and $\ker(\pi)$ in place of $(f_1, \dots, f_s)$.
  This way, we construct a functor
  \begin{equation*}
    J: \{\text{algebras}\} \to \{\text{differential algebras}\},
  \end{equation*}
  which is left adjoint to the forgetful functor $\{\text{differential algebras}\} \to \{\text{algebras}\}$.
\end{remark}

Given an algebra $R$, it is useful to consider the functor
\begin{equation*}
  JR \otimes_R \bullet: \text{$R$-Mod} \to \text{$JR$-Mod},
\end{equation*}
where $JR$ is merely considered an algebra.
Again, the $JR$-module $M$ together with the $R$-module inclusion $\inc: M \hookrightarrow JR \otimes_R M, \inc(u) = 1\otimes u$ satisfy a universal property similar to that of $\inc: R \hookrightarrow JR$, and the functor $JR \otimes_R \bullet$ is left adjoint to the forgetful functor $\text{$JR$-Mod} \to \text{$R$-Mod}$.

Let $R$ be a graded algebra with Hamiltonian $H$.
We can extend uniquely the Hamiltonian $H$ to a Hamiltonian $H^{JR} \in \End(JR)$ because of \eqref{eq:13}.
Furthermore, if $M$ is a graded $R$-module with Hamiltonian $H^M$, we can define a Hamiltonian of $JR \otimes_R M$ by setting
\begin{equation*}
  H^{JR \otimes_R M} = H^{JR}\otimes\Id_{M} + \Id_{JR}\otimes H^M.
\end{equation*}

\bibliographystyle{alpha}
\bibliography{boundary-minimal-models}

\end{document}