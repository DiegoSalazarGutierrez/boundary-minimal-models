\documentclass[a4paper, 12pt, reqno]{amsart}

\usepackage[margin = 0.75in]{geometry}
\usepackage{amssymb}

\DeclareMathOperator{\Vir}{Vir}
\DeclareMathOperator{\ch}{ch}
\DeclareMathOperator{\len}{len}
\DeclareMathOperator{\vac}{|0\rangle}
\DeclareMathOperator{\gr}{gr}
\DeclareMathOperator{\Hom}{Hom}
\DeclareMathOperator{\vspan}{span}

\begin{document}

\title{Boundary minimal models}
\author{Diego Salazar}
\address{Instituto de Matemática Pura e Aplicada, Rio de Janeiro, RJ, Brazil}
\email{diego.salazar@impa.br}
\date{\today}
\maketitle

We set:
\begin{align*}
  c_{p, q} &= 1 - \frac{6(p - q)^2}{pq} \quad \text{for $p, q \ge 2$ relatively prime integers}, \\
  h_{m, n} &= \frac{(np - mq)^2 - (p - q)^2}{4pq} \quad \text{for $0 < m < p$ and $0 < n < q$.}
\end{align*}
Then
\begin{align*}
  \ch_{L(c_{p, q}, h_{m, n})}(q) &= \frac{1}{(q)_{\infty}}\sum_{k \in \mathbb{Z}}q^{\frac{(2kpq + mq - np)^2 - (p - q)^2}{4pq}} - q^{\frac{(2kpq + mq + np)^2 - (p - q)^2}{4pq}} \\
                                 &= \frac{q^{h_{m, n}}}{(q)_{\infty}}\sum_{k \in \mathbb{Z}}q^{k^2pq + k(mq - np)}-q^{k^2pq + k(mq + np) + mn} \\
                                 &= \frac{q^{h_{m, n}}}{(q)_{\infty}}\sum_{k \in \mathbb{Z}}q^{k^2pq + k(mq - np)}-q^{(m + kp)(n + kq)}.
\end{align*}
One can also replace $m \mapsto p - m$ and $n \mapsto q - n$, but I don't think this is useful.

In this sketch, we take:
\begin{align*}
  p &= 2, \\
  q &= 2s + 1 \quad \text{for $s \in \mathbb{Z}_+$}.
\end{align*}
Thus, $\Vir_{2, 2s + 1}$ has $s$ modules $L(c_{2, 2s + 1}, h_{1, 1}), L(c_{2, 2s + 1}, h_{1, 2}), \dots, L(c_{2, 2s + 1}, h_{1, s})$.

NOTE: In the following computations, we take $\len(L_{-2}) = \len(L_{-3}) = \dots = 2$ and $\len(L_{-1}) = 1$ because we will deal with modules.

\section{$s = 1, \Vir_{2, 3}$}
\label{sec:s-=-1}

\begin{equation*}
  p^0(t, q) = 1.
\end{equation*}

\section{$s = 2, \Vir_{2, 5}$}
\label{sec:vir_2-5-1}

\subsection{$m = 1, n = 1$}
\label{sec:m-=-1}

\begin{align*}
  R^{h_{1, 1}} &= R^0 = \{L_{t + 1}^{2 - k}L_t^k \mid t \ge 2, 0 \le k \le 1\}, \\
  p^0(t, q) &= p^0_{>2}(t, q) + p^0_{>2, 2}(t, q), \\
  p^0(t, q) &= \sum_{k \in \mathbb{N}}t^{2k}\frac{q^{k^2 + k}}{(q)_k}, \\
  p^0_{>2}(t, q) &= \sum_{k \in \mathbb{N}}t^{2k}\frac{q^{k^2 + 2k}}{(q)_k}, \\
  p^0_{>2, 2}(t, q) &= \sum_{k \in \mathbb{N}}t^{2k + 2}\frac{q^{k^2 + 3k + 2}}{(q)_k}.
\end{align*}

\subsection{$m = 1, n = 2$}
\label{sec:m-=-1-1}

NOTE: $p_{1^i}$ means the series of the partitions ending in exactly $i$ ones.
\begin{align*}
  R^{h_{1, 2}} &= R^0 \cup \{[2], [1, 1, 1], [3, 1, 1]\}, \\
  p(t, q) &= p_{1^0}(t, q) + p_{1^1}(t, q) + p_{1^2}(t, q), \\
  p_{1^0}(t, q) &= p^0_{>2}(t, q) = \sum_{k \in \mathbb{N}}t^{2k}\frac{q^{k^2 + 2k}}{(q)_k}, \\
  p_{1^1}(t, q) &= tqp^0_{>2}(t, q) = \sum_{k \in \mathbb{N}}t^{2k + 1}\frac{q^{k^2 + 2k + 1}}{(q)_k}, \\
  p_{1^2}(t, q) &= t^2q^2p^0_{>2}(tq^{1/2}, q) = \sum_{k \in \mathbb{N}}t^{2k + 2}\frac{q^{k^2 + 3k + 2}}{(q)_k}.
\end{align*}

\section{$s = 3, \Vir_{2, 7}$}
\label{sec:s-=-3}

\subsection{$m = 1, n = 1$}
\label{sec:m-=-1-2}

\begin{align*}
  R^{h_{1, 1}} &= R^0 = \{L_{t + 1}^{3 - k}L_t^k \mid t \ge 2, 0 \le k \le 2\}, \\
  p^0(t, q) &= p^0_{2, 2}(t, q) + p^0_{>2, 2}(t, q) + p^0_{>2}(t, q), \\
  p^0(t, q) &= \sum_{k_1, k_2 \in \mathbb{N}}t^{2k_1 + 4k_2}\frac{q^{k_1^2 + 2k_1k_2 + 2k_2^2 + k_1 + 2k_2}}{(q)_{k_1}(q)_{k_2}}, \\
  p^0_{>2}(t, q) &= \sum_{k_1, k_2 \in \mathbb{N}}t^{2k_1 + 4k_2}\frac{q^{k_1^2 + 2k_1k_2 + 2k_2^2 + 2k_1 + 4k_2}}{(q)_{k_1}(q)_{k_2}}, \\
  p^0_{>2, 2}(t, q) &= \sum_{k_1, k_2 \in \mathbb{N}}t^{2k_1 + 4k_2 + 2}\frac{q^{k_1^2 + 2k_1k_2 + 2k_2^2 + 2k_1 + 5k_2 + 2}}{(q)_{k_1}(q)_{k_2}}, \\
  p^0_{2, 2}(t, q) &= \sum_{k_1, k_2 \in \mathbb{N}}t^{2k_1 + 4k_2 + 4}\frac{q^{k_1^2 + 2k_1k_2 + 2k_2^2 + 3k_1 + 6k_2 + 4}}{(q)_{k_1}(q)_{k_2}}.
\end{align*}

\subsection{$m = 1, n = 2$}
\label{sec:m-=-1-3}

\begin{align*}
  R^{h_{1, 2}} &= R^0 \cup \{[2], [1, 1, 1, 1, 1], [3, 1, 1, 1, 1], [3, 3, 1, 1]\}, \\
  p(t, q) &= p_{1^0}(t, q) + p_{1^1}(t, q) + p_{1^2}(t, q) + p_{1^3}(t, q) + p_{1^4}(t, q), \\
  p_{1^0}(t, q) &= p^0_{>2}(t, q), \\
  p_{1^1}(t, q) &= tqp^0_{>2}(t, q), \\
  p_{1^2}(t, q) &= t^2q^2(p^0_{>2}(tq^{1/2}, q) + p^0_{>2, 2}(tq^{1/2}, q)), \\
  p_{1^3}(t, q) &= t^3q^3(p^0_{>2}(tq^{1/2}, q) + p^0_{>2, 2}(tq^{1/2}, q)), \\
  p_{1^4}(t, q) &= t^4q^4p^0_{>2}(tq^{1/2}, q).
\end{align*}

\subsection{$m = 1, n = 3$}
\label{sec:m-=-1-4}

\begin{align*}
  R^{h_{1, 3}} &= R^0 \cup \{[2, 1], [2, 2], [1, 1, 1, 1, 1], [3, 1, 1, 1, 1], [3, 3, 1, 1, 1]\}, \\
  p(t, q) &= p_{1^0}(t, q) + p_{1^1}(t, q) + p_{1^2}(t, q) + p_{1^3}(t, q) + p_{1^4}(t, q), \\
  p_{1^0}(t, q) &= p^0_{>2}(t, q) + p_{>2, 2}(t, q), \\
  p_{1^1}(t, q) &= tqp^0_{>2}(t, q), \\
  p_{1^2}(t, q) &= t^2q^2p^0_{>2}(t, q), \\
  p_{1^3}(t, q) &= t^3q^3(p^0_{>2}(tq^{1/2}, q) + p^0_{>2, 2}(tq^{1/2}, q)), \\
  p_{1^4}(t, q) &= t^4q^4p^0_{>2}(tq^{1/2}, q).
\end{align*}

\section{$\Vir_{2, 9}$}
\label{sec:vir_2-9}

\begin{align*}
  p(t, q) &= \sum_{k_1, k_2, k_3 \in \mathbb{N}}t^{k_1 + 2k_2 + 3k_3}\frac{q^{k_1^2 + 2k_1k_2 + 2k_1k_3 + 2k_2^2 + 4k_2k_3 + 3k_3^2 + k_1 + 2k_2 + 3k_3}}{(q)_{k_1}(q)_{k_2}(q)_{k_3}}, \\
  p_{>2}(t, q) &= \sum_{k_1, k_2, k_3 \in \mathbb{N}}t^{k_1 + 2k_2 + 3k_3}\frac{q^{k_1^2 + 2k_1k_2 + 2k_1k_3 + 2k_2^2 + 4k_2k_3 + 3k_3^2 + 2k_1 + 4k_2 + 6k_3}}{(q)_{k_1}(q)_{k_2}(q)_{k_3}}, \\
  p_{>2, 2}(t, q) &= \sum_{k_1, k_2, k_3 \in \mathbb{N}}t^{k_1 + 2k_2 + 3k_3 + 1}\frac{q^{k_1^2 + 2k_1k_2 + 2k_1k_3 + 2k_2^2 + 4k_2k_3 + 3k_3^2 + 2k_1 + 4k_2 + 7k_3 + 2}}{(q)_{k_1}(q)_{k_2}(q)_{k_3}}, \\
  p_{>2, 2, 2}(t, q) &= \sum_{k_1, k_2, k_3 \in \mathbb{N}}t^{k_1 + 2k_2 + 3k_3 + 2}\frac{q^{k_1^2 + 2k_1k_2 + 2k_1k_3 + 2k_2^2 + 4k_2k_3 + 3k_3^2 + 2k_1 + 5k_2 + 8k_3 + 4}}{(q)_{k_1}(q)_{k_2}(q)_{k_3}} \\
  p_{2, 2, 2}(t, q) &= \sum_{k_1, k_2, k_3 \in \mathbb{N}}t^{k_1 + 2k_2 + 3k_3 + 3}\frac{q^{k_1^2 + 2k_1k_2 + 2k_1k_3 + 2k_2^2 + 4k_2k_3 + 3k_3^2 + 3k_1 + 6k_2 + 9k_3 + 6}}{(q)_{k_1}(q)_{k_2}(q)_{k_3}}.
\end{align*}

\section{Conjecture $\Vir_{2, 2s + 1}$}
\label{sec:conjecture}

\begin{align*}
  p(t, q) = \sum_{k = (k_1, \dots, k_{s - 1}) \in \mathbb{N}^{s - 1}}t^{2kB^{(s)}}\frac{q^{\frac{1}{2}kG^{(s)}k^T + kB^{(s)}}}{(q)_{k_1}\dots (q)_{k_{s - 1}}},
\end{align*}
where
\begin{align*}
  G^{(s)} &= (2\min\{i, j\})_{i, j = 1}^{s - 1}, \\
  B^{(s)} &=
            \left(\begin{smallmatrix}
                    1 \\
                    2 \\
                    \vdots \\
                    s - 1
                  \end{smallmatrix}\right).
\end{align*}

\section{New conjecture $\Vir_{2, 2s + 1}$}
\label{sec:new-conjecture-vir_2}

NOTE: This is probably proven in Andrew's book The theory of partitions.
The objective is to prove

\begin{equation*}
  p(t, q) = \sum_{i = 0}^{s - 1}p_{>2, \underbrace{\scriptstyle{2, \dots, 2}}_i}(t, q) = \sum_{k = (k_1, \dots, k_{s - 1}) \in \mathbb{N}^{s - 1}}t^{2kB^{(s)}_{s - 1}}\frac{q^{\frac{1}{2}kG^{(s)}k^T + kB^{(s)}_{s - 1}}}{(q)_{k_1}\dots (q)_{k_{s - 1}}},
\end{equation*}
where
\begin{align*}
  p_{>2, \underbrace{\scriptstyle{2, \dots, 2}}_i}(t, q) &= \sum_{k = (k_1, \dots, k_s) \in \mathbb{N}^{s - 1}}t^{2kB^{(s)}_{s - 1} + 2i}\frac{q^{\frac{1}{2}kG^{(s)}k^T + k(2B^{(s)}_{s - 1} + B^{(s)}_i) + 2i}}{(q)_{k_1}\dots(q)_{k_{s - 1}}}, \quad i = 0, 1, \dots, s - 1, \\
  G^{(s)} &= (2\min\{i, j\})_{i, j = 1}^{s - 1}, \\
  B^{(s)}_i &=
              \left(\begin{smallmatrix}
                      0 \\
                      0 \\
                      \vdots \\
                      0 \\
                      1 \\
                      2 \\
                      \vdots \\
                      i
                    \end{smallmatrix}\right), \quad i = 0, 1, \dots, s - 1.  
\end{align*}

These series should satisfy the following (though I can't prove the first set of equalities):
\begin{align*}
  p_{>2, \underbrace{\scriptstyle{2, \dots, 2}}_i}(t, q) &= t^{2i}q^{2i}\sum_{j = 0}^{s - 1 - i}p_{>2, \underbrace{\scriptstyle{2, \dots, 2}}_j}(tq, q), \quad i = 0, 1, \dots, s - 1, \\
  p_{>2, \underbrace{\scriptstyle{2, \dots, 2}}_i}(0, 0) &= \delta_{0, i}, \quad i = 0, 1, \dots, s - 1.
\end{align*}

\section{New New conjecture}
\label{sec:new-new-conjecture}

Now we take $\len(L_{-1}) = \len(L_{-2}) = \dots = 1$ so the result is NOT the same.

\begin{align*}
  \ch_{L(c_{2, 2s + 1}, h_{1, i})}(t, q) &= \sum_{j = 0}^{i - 1}p_{1^j}(t, q), \quad i = 1, \dots, s, \\
  p_{1^j}(t, q) &= \sum_{k = (k_1, \dots, k_s) \in \mathbb{N}^{s - 1}}t^{kB^{(s)}_{s - 1} + j}\frac{q^{\frac{1}{2}kG^{(s)}k^T + k(B^{(s)}_{s - 1} + B^{(s)}_j) + j}}{(q)_{k_1}\dots(q)_{k_{s - 1}}}, \quad j = 0, 1, \dots, s - 1,
\end{align*}
where
\begin{align*}
  G^{(s)} &= (2\min\{i, j\})_{i, j = 1}^{s - 1}, \\
  B^{(s)}_j &=
              \left(\begin{smallmatrix}
                      0 \\
                      0 \\
                      \vdots \\
                      0 \\
                      1 \\
                      2 \\
                      \vdots \\
                      j
                    \end{smallmatrix}\right), \quad j = 0, 1, \dots, s - 1.  
\end{align*}

\section{Context, work done by Andrews and what we are trying to calculate}
\label{sec:context-work-done}

In this section, we explain what we are trying to calculate, what was done by Andrews and show the difference between the vertex algebra module approach and the lie algebra module approach.

Note that in the case $i = 1$, we have $h_{1, 1} = 0$, and we are in the vertex algebra case $L(c_{2, 2s + 1}, 0) = \Vir_{c_{2, 2s + 1}}$.
This case is studied in the book The theory of partitions, as we now show.
We first recall that $R_{\Vir^{2, 2s + 1}} = \mathbb{C}[L_{-2}]$.
We know that $\Vir_{2, 2s + 1}$ is classically free, and $R_{\Vir_{2, 2s + 1}} = \mathbb{C}[L_{-2}]/(L_{-2}^s)$ because the maximal ideal of $\Vir^{2, 2s + 1}$ is generated by an element $b_s$ with leading monomial $L_{-2}^s\vac$ in the degree reverse lexicographical order.

Thus, we have 
\begin{equation*}
  JR_{\Vir_{2, 2s + 1}}/(L_{-2}^s)_{\partial} = \mathbb{C}[L_{-2}, L_{-3}, \dots]/(L_{-2}^s)_{\partial}.
\end{equation*}

The leading term ideal of $(L_{-2}^s)$ in $\mathbb{C}[L_{-2}, L_{-3}, \dots]$ is generated by the polynomials $L_{-\lambda_1} \dots L_{-\lambda_m} \in \mathbb{C}[L_{-2}, L_{-3}, \dots]$, where $\lambda = [\lambda_1, \dots, \lambda_m]$ is a partition satisfying $\lambda_m \ge 2$ and the difference condition
\begin{equation*}
  \lambda_i - \lambda_{i + s - 1} \le 1 \quad \text{for some $i$ with $1 \le i \le m + 1 - s$}.
\end{equation*}
Therefore, by Grobner basis theory, a basis of $JR_{\Vir_{2, 2s + 1}}/(L_{-2}^s)_{\partial}$ is given by classes of polynomials $L_{-\lambda_1} \dots L_{-\lambda_m} \in \mathbb{C}[L_{-2}, L_{-3}, \dots]$, where $\lambda = [\lambda_1, \dots, \lambda_m]$ is a partition satisfying $\lambda_m \ge 2$ and the difference condition
\begin{equation}
  \label{eq:1}
  \lambda_i - \lambda_{i + s - 1} \ge 2 \quad \text{for $1 \le i \le m + 1 - s$}.
\end{equation}
NOTE: Actually, this is handwaving a little bit because strictly speaking, Grobner basis theory only applies to finitely generated polynomial algebras.
But we can make this rigorous by studying $\mathbb{C}[L_{-2}, L_{-3}, \dots, L_{-N}]$ and then letting $N \to \infty$.

In Theorem 7.5 of The theory of partitions, the number of partitions $\lambda$ of $n$ satisfying $\lambda_m \ge 2$ and \eqref{eq:1} is denoted by $B_{s, 0}$ (note that we are changing $k$ to $s$ here and $i = 0$).
In that section, it is also defined by $b_{s, 0}(m, n)$ the number of partitions $\lambda$ of $n$ with exactly $m$ parts satisfying $\lambda_m \ge 2$ and \eqref{eq:1}.
It is also noted that $b_{s, 0}(m, n) = c_{s, 0}(m, n)$, where
\begin{equation*}
  J_{s, 0}(0; t; q) = \sum_{m = 0}^{\infty}\sum_{n = 0}^{\infty}c_{s, 0}(m, n)t^mq^n,
\end{equation*}
and we have changed $x$ to $t$.

Note that, using the notation of Theorem 7.8 and the previous section, we have
\begin{align*}
  N_1^2 + \dots + N_{s - 1}^2 &= \frac{1}{2}kG^{(s)}k^T, \\
  N_1 + \dots + N_{s - 1} &= kB^{(s)}_{s - 1},
\end{align*}
where we have replaced $k$ by $s$.
By (the proof of) Theorem 7.8, we have
\begin{equation*}
  J_{s, 0}(0; t; q) = \sum_{k = (k_1, \dots, k_s) \in \mathbb{N}^{s - 1}}t^{kB^{(s)}_{s - 1}}\frac{q^{\frac{1}{2}kG^{(s)}k^T + kB^{(s)}_{s - 1}}}{(q)_{k_1}\dots(q)_{k_{s - 1}}}.
\end{equation*}

From the natural isomorphisms
\begin{align*}
  JR_{\Vir_{2, 2s + 1}}/(L_{-2}^s)_{\partial} &\cong \gr_F(\Vir_{2, 2s + 1}), \\
  \gr_F(\Vir_{2, 2s + 1}) &\cong \Vir_{2, 2s + 1},
\end{align*}
we have that a basis of $\Vir_{2, 2s + 1}$ is given by the classes of elements of the form $L_{-\lambda_1}\dots L_{-\lambda_m}\vac$, where $\lambda$ is a partition satisfying $\lambda_m \ge 2$ and \eqref{eq:1}.

In conclusion, the refined character of $\Vir_{2, 2s + 1}$ is given by 
\begin{align*}
  \ch_{\Vir_{2, 2s + 1}}(t, q) &= \sum_{m = 0}^{\infty}\sum_{n = 0}^{\infty}b_{s, 0}(m, n)t^mq^n = \sum_{m = 0}^{\infty}\sum_{n = 0}^{\infty}c_{s, 0}(m, n)t^mq^n \\
                               &= J_{s, 0}(0; t; q) = \sum_{k = (k_1, \dots, k_s) \in \mathbb{N}^{s - 1}}t^{kB^{(s)}_{s - 1}}\frac{q^{\frac{1}{2}kG^{(s)}k^T + kB^{(s)}_{s - 1}}}{(q)_{k_1}\dots(q)_{k_{s - 1}}},
\end{align*}
and this is precisely the formula in the previous section with $i = 0$.

NOTE 2: I'm not sure we can use other values of $i$ other than $i = 0$.
But probably, we can.
VERY IMPORTANT NOTE: We have not defined what is $\ch_M(t, q)$, where $M$ is a $\Vir$-module.
And we have not calculated the refined character with respect to standard filtration $G^p\Vir_{2, 2s + 1}$ because there is a missing factor $2$ coming from the conformal weight of $L_{-2}\vac$ being $2$.
Sure, we could simply add it here and put $t^{2kB^{(s)}}$ instead of $t^{kB^{(s)}}$.
But the situation is different when we deal with modules because in that case $L_{-1}$ has length $1$ while $L_{-2}, L_{-3}, \dots$ have length $2$ and then the resulting character is probably not given by simply multiplying by $2$ or other trivial operation.
Thus, we have to truly define properly the meaning of the refined character in the case of $\Vir$-modules.
Sadly, if we do it like this, this does not have much to do with modules over vertex algebras and probably we can't study classically free modules over vertex algebras (or some concepts related to that).
But when I tried calculating the refined characters using the standard filtration, I could not find a closed formula.
And the results in Andrew's book probably don't help much if we use the standard filtration of modules over vertex algebras.
Thus, in the next section, we will define the refined character of $\Vir$-modules.

\section{The refined character}
\label{sec:refined-character}

Let $\Gamma$ be an abelian group.
For a $\Gamma$-graded vector space $V = \bigoplus_{\alpha \in \Gamma}V^{\alpha}$, we set $\mathcal{P}(V) = \{\alpha \in \Gamma \mid V^{\alpha} \neq 0\}$.

A \emph{$\Gamma$-graded Lie algebra} is a Lie algebra $\mathfrak{g} = \bigoplus_{\mathfrak{g} \in \Gamma}\mathfrak{g}^{\alpha}$ such that
\begin{equation*}
  [\mathfrak{g}^{\alpha}, \mathfrak{g}^{\beta}] \subseteq \mathfrak{g}^{\alpha + \beta} \quad \text{for $\alpha, \beta \in \Gamma$}.
\end{equation*}

Let $Q$ be a free abelian group of finite rank $r$, and let $\mathfrak{g}$ be a Lie algebra with a commutative subalgebra $\mathfrak{h}$.
We say a pair $(\mathfrak{g}, \mathfrak{h})$ is a \emph{$Q$-graded Lie algebra} if it satisfies the following:
\begin{enumerate}
\item $\mathfrak{g} = \bigoplus_{\alpha \in Q}\mathfrak{g}^{\alpha}$ is $Q$-graded, $\mathfrak{h} = \mathfrak{g}^0$, and $\mathcal{P}(\mathfrak{g})$ generates $Q$;
\item We have a homomorphism $\pi_Q: Q \to \mathfrak{h}^*, \alpha \mapsto \lambda_{\alpha}$ such  that
  \begin{equation*}
    [h, x] = \lambda_{\alpha}(h)x \quad \text{for $h \in \mathfrak{h}$ and $x \in \mathfrak{g}^{\alpha}$};
  \end{equation*}
\item For $\alpha \in Q$, $\dim(\mathfrak{g}^{\alpha}) < \infty$;
\item There exists a basis $(\alpha_i)_{i = 1}^r$ of $Q$ such that for $\alpha \in Q$ with $\mathfrak{g}^{\alpha} \neq 0$,
  \begin{equation*}
    \text{$\alpha \in \sum_{i = 1}^r\mathbb{Z}_{\ge 0}\alpha_i$ or $\alpha \in \sum_{i = 1}^r\mathbb{Z}_{\le 0}\alpha_i$}.
  \end{equation*}
\end{enumerate}

The condition (iv) implies that a $Q$-graded Lie algebra admits a \emph{triangular decomposition}.
  If we set $Q^+ = \sum_{i = 1}^r\mathbb{Z}_{\ge 0}\alpha_i$ and
  \begin{equation*}
    \mathfrak{g}^{\pm} = \bigoplus_{\pm \alpha \in Q^+ \setminus \{0\}}\mathfrak{g}^{\alpha},
  \end{equation*}
  then we have $\mathfrak{g} = \mathfrak{g}^- \oplus \mathfrak{h} \oplus \mathfrak{g}^+$.
  For later use, we set $\mathfrak{g}^{\ge} = \mathfrak{h} \oplus \mathfrak{g}^+$ and $\mathfrak{g}^{\le} = \mathfrak{g}^- \oplus \mathfrak{h}$.

The \emph{Virasoro Lie algebra}, denoted by $\Vir$, is the Lie algebra given by:
  \begin{align*}
    \Vir &= \bigoplus_{n \in \mathbb{Z}}\mathbb{C}L_n \oplus \mathbb{C}C, \\
    [L_m, L_n] &= (m - n)L_{m + n} + \delta_{m, -n}\frac{m^3 - m}{12}C \quad \text{for $m, n \in \mathbb{Z}$}, \\
    [\Vir, C] &= 0.
  \end{align*}

  We set $Q = \mathbb{Z}$, $\mathfrak{h} = \mathbb{C}L_0 \oplus \mathbb{C}C$ and
  \begin{align*}
    \pi_Q: Q &\to \mathfrak{h}^*, \\
    \pi_Q(n) &= (L_0 \mapsto -n, C \mapsto 0) \quad \text{for $n \in \mathbb{Z}$}.
  \end{align*}
  Then $(\Vir, \mathfrak{h})$ is readily seen to be a $\mathbb{Z}$-graded Lie algebra.

Let $\Gamma$ be an abelian group, and let $\mathfrak{g} = \bigoplus_{\alpha \in \Gamma}\mathfrak{g}^{\alpha}$ be a $\Gamma$-graded Lie algebra.
  A $\mathfrak{g}$-module $M = \bigoplus_{\alpha \in \Gamma}M^{\alpha}$ is \emph{$\Gamma$-graded} if
  \begin{equation*}
    \mathfrak{g}^{\alpha}M^{\beta} \subseteq M^{\alpha + \beta} \quad \text{for $\alpha, \beta \in \Gamma$}.
  \end{equation*}

An $\mathfrak{h}$-module $M$ is \emph{$\mathfrak{h}$-diagonalizable} if
  \begin{equation*}
    M = \bigoplus_{\lambda \in \mathfrak{h}^*}M_{\lambda},
  \end{equation*}
  where $M_{\lambda} = \{v \in M \mid \text{for $h \in \mathfrak{h}$, $hv = \lambda(h)v$}\}$.
  
An $\mathfrak{h}$-diagonalizable module $M$ is called \emph{$\mathfrak{h}$-semisimple} if
  \begin{equation*}
    \dim(M_{\lambda}) < \infty \quad \text{for $\lambda \in \mathfrak{h}^*$}.
  \end{equation*}
  A $(\mathfrak{g}, \mathfrak{h})$-module is an $\mathfrak{h}$-diagonalizable $\mathfrak{g}$-module.
  We assume that $\pi_Q$ is injective.

  The \emph{character} of a semisimple $(\mathfrak{g}, \mathfrak{h})$-module $M$ is given by
  \begin{equation*}
    \ch_M(q) = \sum_{\lambda \in \mathfrak{h}^*}\dim(M_{\lambda})q^{\lambda}.
  \end{equation*}

We can regard $\mathfrak{g}$ as a $(\mathfrak{g}, \mathfrak{h})$-module via the adjoint representation and by declaring for $\lambda \in \mathfrak{h}^*$,
\begin{equation*}
  \mathfrak{g}_{\lambda} = \mathfrak{g}^{\alpha} \quad \text{if exists $\alpha \in Q$ such that $\lambda = \pi_Q(\alpha)$}.
\end{equation*}

The category $\mathcal{C}_{(\mathfrak{g}, \mathfrak{h})}$ is the category whose objects are $\mathfrak{h}^*$-graded $(\mathfrak{g}, \mathfrak{h})$-modules and for $M, N \in \mathcal{C}_{(\mathfrak{g}, \mathfrak{h})}$, $\Hom_{\mathcal{C}_{(\mathfrak{g}, \mathfrak{h})}}(M, N) = \Hom_{\mathfrak{g}}(M, N)$ (lie algebra homomorphisms).

Let $(\mathfrak{g}, \mathfrak{h})$ be a $Q$-graded Lie algebra.
  Let $M \in \mathcal{C}_{(\mathfrak{g}, \mathfrak{h})}$, and let $\lambda \in \mathfrak{h}^*$.
  $M$ is called a \emph{highest weight module} with \emph{highest weight} $\lambda \in \mathfrak{h}^*$ if there is a nonzero vector $v \in M_{\lambda}$ such that:
  \begin{enumerate}
  \item $xv = 0$ for $x \in \mathfrak{g}^+$;
  \item $U(\mathfrak{g}^-)v = M$.
  \end{enumerate}
  The vector $v$ is called a \emph{highest weight vector} of $M$ and is unique up to multiplication by a nonzero scalar.

  When considering the $\mathbb{Z}$-graded Virasoro Lie algebra $(\Vir, \mathfrak{h})$, where $\mathfrak{h} = \mathbb{C}L_0 \oplus \mathbb{C}C$, we identify $\mathfrak{h}^*$ with $\mathbb{C}^2$ as $\lambda = (c, h)$ if $\lambda(C) = c$ and $\lambda(L_0) = h$.

  Let $A$ be an associative (not necessarily commutative) algebra with unit $1$ and filtration $(A^p)_{p \in \mathbb{Z}}$ such that:
\begin{enumerate}
\item $A^p = 0$ for $p < 0$;
\item $1 \in A^0$;
\item $A^0 \subseteq A^1 \subseteq \dots$;
\item $A^pA^q \subseteq A^{p + q}$ for $p, q \in \mathbb{Z}$.
\end{enumerate}
Let
\begin{equation*}
  \gr(A) = \bigoplus_{p \in \mathbb{N}}A^p/A^{p - 1}
\end{equation*}
be the associated graded vector space.
The vector space $\gr(A)$ is an associative algebra with unit and multiplication given as follows.
For $p, q \in \mathbb{N}$, $a \in A^p$ and $b \in A^p$, we set
\begin{equation*}
  \gamma^p(a)\gamma^q(b) = \gamma^{p + q}(ab),
\end{equation*}
where $\gamma^p: A^p \to \gr(A)$ is the \emph{principal symbol map}, which is the composition of the natural maps $A^p \twoheadrightarrow A^p/A^{p - 1}$ and $A^p/A^{p - 1} \hookrightarrow \gr(A)$.
The unit of $\gr(A)$ is $\gamma^0(1)$.

Let $\mathfrak{g}$ be a Lie algebra.
The \emph{PBW filtration of $U(\mathfrak{g})$}, the universal enveloping algebra of $\mathfrak{g}$, is given by
\begin{equation*}
  U(\mathfrak{g})^p = \vspan\{x_1x_2\dots x_s \mid s \le p, x_1, \dots, x_s \in \mathfrak{g}\} \quad \text{for $p \in \mathbb{Z}$}.
\end{equation*}
This filtration clearly satisfies axioms (i)--(iv) above, and $\gr(U(\mathfrak{g}))$ is naturally isomorphic to $S(\mathfrak{g})$, the symmetric algebra of $\mathfrak{g}$, which is a polynomial algebra.

Let $M$ be an $A$-module with filtration $(M^p)_{p \in \mathbb{Z}}$ such that:
\begin{enumerate}
\item $M^p = 0$ for $p < 0$;
\item $M^0 \subseteq M^1 \subseteq \dots$;
\item $A^pM^q \subseteq M^{p + q}$ for $p, q \in \mathbb{Z}$.
\end{enumerate}
Let
\begin{equation*}
  \gr(M) = \bigoplus_{p \in \mathbb{N}}M^p/M^{p - 1}
\end{equation*}
be the associated graded vector space.

Then $\gr(M)$ is a $\gr(A)$-module with operations given as follows.
For $p, q \in \mathbb{N}$, $a \in A^p$ and $u \in M^p$, we set
\begin{equation*}
  \gamma^p(a)\gamma^q_M(u) = \gamma^{p + q}_M(au),
\end{equation*}
where $\gamma^p_M: M^p \to \gr(M)$ is the \emph{principal symbol map}, which is the composition of the natural maps $M^p \twoheadrightarrow M^p/M^{p - 1}$ and $M^p/M^{p - 1} \hookrightarrow \gr(M)$.

Let $(\mathfrak{g}, \mathfrak{h})$ be a $Q$-graded Lie algebra, and let $M$ be a semisimple highest weight module with highest weight $\lambda \in \mathfrak{h}^*$ and highest weight vector $v \in M_{\lambda}$.
The \emph{PBW filtration of $M$} is given by
\begin{equation*}
  M^p = U(\mathfrak{g}^-)^pv \quad \text{for $p \in \mathbb{Z}$}.
\end{equation*}
This filtration clearly satisfies axioms (i)--(iii) above with $U(\mathfrak{g}^-)$ in place of $A$, and $\gr(M)$ becomes a $\gr(U(\mathfrak{g^-}))$-module.
Actually, $\gr(M)$ is naturally isomorphic to a quotient of $\gr(U(\mathfrak{g}^-))$, so we are in the polynomial case only, at least in what we are interested now.

The \emph{refined character of $M$} is defined by
\begin{equation*}
  \ch_M(t, q) = \sum_{p \in \mathbb{N}}\sum_{\lambda \in \mathfrak{h}^*}\dim(\gamma^p(U(\mathfrak{g}^-)^pv \cap M_{\lambda}))t^pq^{\lambda},
\end{equation*}
and this is what we are interested in.

EXAMPLE: We can consider the $\mathbb{Z}$-graded Virasoro Lie algebra $(\Vir, \mathfrak{h})$.
We are mainly interested in $M(c, h)$ and $L(c, h)$.
For example,
\begin{equation*}
  \ch_{M(c, h)}(t, q) = \frac{q^h}{\prod_{k \in \mathbb{Z}_+}(1 - tq^k)}.
\end{equation*}

\end{document}
